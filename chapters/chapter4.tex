%************************************************
\chapter{Local algorithm}\label{ch:local}
%************************************************

In this Chapter will be defined a local Metropolis-Hastings algorithm that samples link variables.
%Then, a specific local algorithm will be defined to compute the integral of Equation \eqref{eq:lat_exp}.
The first simulations will be performed,
and the result of the plaquette mean energy will be compared with the value reported by D\"urr and Hoelbling in 2005 \cite{durr-hoelbling:2005}.

Even though the acceptance of the local algorithm will be close to its possible maximum value of one,
since the algorithm is local, at small lattice spacing, the topological freezing will be encountered.

This will cause both the divergence of the charge autocorrelation time,
and the biasedness of observables mean values.

\section{Link variables updating}
\subsection*{Heat-bath}
%\begin{align*}
%    S &= \beta\sum_s\left(1-\Re\,U[\mathcal P(s)]\right) \\
%    \Delta S &= -\beta\Re\,\left(U[\mathcal S_0]U[\mathcal L] + U[\mathcal S_1]U[\mathcal L]\right) \\
%             &= -\beta\Re\,(Wu) \\
%             &= -\beta|W|\cos(\phi + \arg W) \\
%             &= e^{-k\cos(x)}
%\end{align*}
The probability density function of links configurations $\{U\}$ is:
\[
    p(U) \propto e^{-S}
\]
in which the action $S$ is:
\[
    S = \beta\sum_s\left(1-\Re\,U[\mathcal P(s)]\right)
\]
Every plaquette value $\Re\,U[\mathcal P(s)]$ depends on four links variables,
and every link variable contributes to two different plaquettes.

Links variables are then simultaneously coupled in their probability density function,
which is then complicated and impossible to sample directly.

However, the action has a local nature,
since every link variable contributes only to the two plaquettes that include it.
Thus, using the heat-bath approach, and considering only the marginal distribution of one link variable,
the contribution of all plaquettes that do not include the link can be integrated out of the distribution.

Let $\mathcal L$ be the link considered, and $\mathcal P_1, \mathcal P_2$ the two plaquettes that include it.
The two paths $\mathcal S_1, \mathcal S_2$ sketched in Figure \ref{fig:staples}
are called \emph{staples}.
They are all the staples connected to $\mathcal L$,
in the sense that they are the staples that form a plaquette if $\mathcal L$ is included to them.

\begin{figure}[!htb]
    \centering
    \begin{tikzpicture}[x=5em,y=5em]
        \draw[step=1, help lines, dashed, color=black!30] (0,0) grid (3.5,2.5);
        \draw[->,thick,black] (0,0) -- (3.5,0) node [right] {$x_1$};
        \draw[->,thick,black] (0,0) -- (0,2.5) node [above] {$x_2$};

        \draw[->-,very thick,orange] (2,1) node [black,below] {$\mathcal L$} --  (2,2);

        %left staple
        \draw[->-,very thick,RoyalBlue] (2,2) -- (1,2);
        \draw[->-,very thick,RoyalBlue] (1,2) -- node [black,left] {$\mathcal S_1$} (1,1);
        \draw[->-,very thick,RoyalBlue] (1,1) -- (2,1);

        %right staple
        \draw[->-,very thick,RoyalBlue] (2,2) -- (3,2);
        \draw[->-,very thick,RoyalBlue] (3,2) -- node [black,right] {$\mathcal S_2$} (3,1);
        \draw[->-,very thick,RoyalBlue] (3,1) -- (2,1);

        %left plaq
        \draw[->-,thick,dashed,black] (1.2,1.2) -- (1.8,1.2);
        \draw[->-,thick,dashed,black] (1.8,1.2) -- (1.8,1.8);
        \draw[->-,thick,dashed,black] (1.8,1.8) -- (1.2,1.8);
        \draw[->-,thick,dashed,black] (1.2,1.8) -- (1.2,1.2);
        \node at (1.5,1.5) {$\mathcal P_1$};

        %right plaq
        \draw[->-,thick,dashed,black] (2.2,1.2) -- (2.8,1.2);
        \draw[->-,thick,dashed,black] (2.8,1.2) -- (2.8,1.8);
        \draw[->-,thick,dashed,black] (2.8,1.8) -- (2.2,1.8);
        \draw[->-,thick,dashed,black] (2.2,1.8) -- (2.2,1.2);
        \node at (2.5,1.5) {$\mathcal P_2$};

        %\node[anchor={north east}] at (1,1) {$\mathcal P_1$};
        %\node[anchor={north east}] at (2,1) {$\mathcal P_2$};

    \end{tikzpicture}
    \caption{Staples connected to a vertical link}
    \label{fig:staples}
\end{figure}

The contribution of $\mathcal P_1, \mathcal P_2$ to the action can be written in terms of
the link variable $u \equiv U[\mathcal L]$ in the following way:
\[\begin{aligned}
    \Re\,U[\mathcal P_1] + \Re\,U[\mathcal P_2] &= \Re\,(uU[\mathcal S_1]) + \Re\,(u^*U^*[\mathcal S_2]) \\
                                                &= \Re\,(Wu)
\end{aligned}\]
where $W \equiv U[\mathcal P_1] + U[\mathcal P_2] \notin U(1)$ has a modulus $|W|\neq1$.
Thus, the marginal distribution of $u$ with all the other link variables kept fixed at $u_\mathrm{fix}$ is:
\[\begin{aligned}
    p(u;u_\mathrm{fix}) &\propto e^{-\beta(1-\Re(Wu))} \\
                        &\propto e^{\beta\Re(Wu)} \\
                        &= e^{\beta|W|\Re\left(u_0\right)}
\end{aligned}\]
Here, $u_0 \equiv \frac{W}{|W|}u \in U(1)$, and $\mathrm du_0=\mathrm du$ are then the same Haar measure of the compact group $U(1)$.
Which means:
\[
    p(u;u_\mathrm{fix}) = p(u_0;u_\mathrm{fix}) \propto e^{\beta|W|\Re\left(u_0\right)}
\]
Indeed, $u$ can be parameterized as $u\equiv e^{i\phi}$ in which $\phi$ is defined in $(-\pi + \phi_0, \pi + \phi_0]$ with a generic $\phi_0$.
After a transformation $u \rightarrow \frac{W}{|W|}u$, $\phi$ undergoes the translation $\phi \rightarrow \phi + \arg W$.
If the parameterization is redefined as $\phi_0 \rightarrow \phi_0 - \arg W$, the shift of the definition interval of $\phi$ is then reabsorbed.

Choosing to parameterize $u_0 \equiv e^{ix}$ with $x \in (-\pi, \pi]$,
the marginal distribution of $x$ finally is:
\begin{equation}\label{eq:local_pdf}
    p(x;u_\mathrm{fix}) \propto e^{k\cos(x)} \quad x \in (-\pi, \pi]
\end{equation}
with $k \equiv \beta|W|$.
Equation \eqref{eq:local_pdf} cannot be integrated analytically,
therefore it is imppossible to directly sample it without numerical approximations.

It was chosen to sample it using a Metropolis-Hastings algorithm,
and, for what has been said at the end of Chapter \ref{ch:mc},
the acceptance of the algorithm increases as much similar $\mathcal P(x_\mathrm{old}\to x_\mathrm{new})$ gets to $p(x_\mathrm{new};u_\mathrm{fix})$.

On this matter, Laplaces's method identifies such a distribution.
Indeed it is its corollary that any unimodal distribution of the form $\propto e^{kf(x)}$ approaches a Gaussian as $k$ increases.

In particular, it proves that (Figure \ref{fig:gauss}):
\begin{equation}\label{eq:gauss_approx}
    e^{k\cos(x)} \stackrel{k\to\infty}{\scalebox{2}[1.25]{$\sim$}} e^{-kx^2/2}
\end{equation}

\begin{figure}[!htb]
    \centering
    \input{gfx/gauss.pgf}
    \caption{$p(x;u_\mathrm{fix})$ converges to a Gaussian as $k$ increases}
    \label{fig:gauss}
\end{figure}

Laplace's method states that, if $f(x)$ is a twice-differentiable function that has only one maximum point $x_0 \in [a,b]$ with $x_0 \neq a,b$,
then:
\begin{equation}\label{eq:laplace}
	\int_a^b\mathrm dx\,e^{kf(x)} \xrightarrow{k\to\infty} e^{kf(x_0)}\int_a^b\mathrm dx\,e^{kf"(x_0)(x-x_0)^2/2}
\end{equation}
Indeed, as $k$ increases, the points that are further from the maximum point $x_0$ are exponentially suppressed,
and only the points in the range of $x_0$ becomes relevant to the integral.
The order zero of the Taylor expansion of $f(x)$ around $x_0$ is a multiplicative constant and the order one does not contribute since it is symmetric.
The meaningful part is then the second order, which is negative since $x_0$ is a maximum,
and hence the integrand of the second term of Equation \eqref{eq:laplace} is a Gaussian.

Applying Laplace's method for the case of Equatiton \eqref{eq:local_pdf},
the global maximum is $x_0=0$, and $f"(x_0) = -1$.
Thus:
\[
    \int_{-\pi}^x\mathrm dx'\,e^{k\cos(x')} \stackrel{k\to\infty}{\scalebox{2}[1.25]{$\sim$}} \int_{-\pi}^x\mathrm dx'\,e^{-k{x'}^2/2} \quad \mathrm{for}\ x>0
\]
and equation \eqref{eq:gauss_approx} is recovered taking the derivative of both members, and extending the results for symmetry around $0$.

\subsection*{Gaussian sampling}
The proposal distribution has then been chosen to be:
\begin{equation}\label{eq:gauss_proposal}
    \mathcal P(y\to x) \propto e^{-kx^2/2} \quad x \in (-\pi,\pi]
\end{equation}

A Gaussian in $(-\infty,\infty)$ can be directly sampled using the Box-Muller method,
and an adaptation of it is needed to restrict the variable into $(-\pi,\pi)$.

The probability density function of two independent random variables $x_1,x_2$ distributed according to $\mathcal P(y\to x)$ is:
\begin{equation}\label{eq:gauss2d}
    p(x_1,x_2) \propto e^{-\frac{k}{2}(x_1^2+x_2^2)}
\end{equation}

Switching to polar coordinates\footnote{The correct extreme values of the interval of definition will be recovered afterwards, adapting the final solution.}:
\[\begin{aligned}
    &\begin{dcases}
        x_1 = r \cos\theta \\
        x_2 = r \sin\theta
    \end{dcases}
    &
    \begin{dcases}
        r \in (0,\pi] \\
        \theta \in [0,2\pi)
    \end{dcases}&
\end{aligned}\]
the distribution in terms of $r$ and $\theta$ becomes:
\[
    p(r,\theta) \propto re^{-\frac{k}{2}r^2}
\]
which is separable into two independent distributions:
\[\begin{aligned}
    p(r,\theta) &= p(r)p(\theta) \\
                &= \left(\mathcal Nre^{-\frac{k}{2}r^2}\right)\left(\frac{1}{2\pi}\right)
\end{aligned}\]
where $\mathcal N$ is a normalization constant.

$p(\theta)$ is a uniform distribution and is then trivial to sample.
To directly sample $p(r)$, instead, can be used the procedure defined in Chapter \ref{ch:mc}.
Indeed, integrating it:
\[
    F(r) = \int_0^r\mathrm dr'\,p(r') = \frac{\mathcal N}{k}\left.e^{-\frac{k}{2}{r'}^2}\right|_r^0 = \frac{\mathcal N}{k}\left(1-e^{-\frac{k}{2}r^2}\right)
\]
and evaluating its inverse $F^{-1}(y_1)$:
\[\begin{aligned}
    y_1 &= F(r) = \frac{\mathcal N}{k}\left(1-e^{-\frac{k}{2}r^2}\right) \\
        &\Rightarrow e^{-\frac{k}{2}r^2} = 1 - y_1\frac{k}{\mathcal N} \\
        &\Rightarrow F^{-1}(y_1) = r = \sqrt{-\frac{2}{k}\log\left(1-\frac{k}{\mathcal N}y_1\right)}
\end{aligned}\]

It remains to evaluate the normalization constant:
\[
    \mathcal N^{-1} = \int_0^\pi\mathrm dr\,re^{-\frac{k}{2}r^2} = \frac{1}{k}\left(1-e^{-\frac{k}{2}\pi^2}\right)
\]
to get:
\[
    F^{-1}(y_1) = \sqrt{-\frac{2}{k}\log\left[1-y_1\left(1-e^{-\frac{k}{2}\pi^2}\right)\right]}
\]
which means that if $y_1, y_2$ are equally distributed in $(0,1)$, then:
\begin{equation*}
    \begin{dcases}
        x_1 = F^{-1}(y_1)\cos(2\pi y_2) \\
        x_2 = F^{-1}(y_1)\sin(2\pi y_2)
    \end{dcases}
\end{equation*}
are two independent variables distributed according to \eqref{eq:gauss2d}.

However they do not span homogeneously all the interval $(-\pi,\pi]$ if the probability of being exactly at points $0$ and $\pi$ is taken into account,
which is non-zero in numerical applications because the representation of numbers in computer memory is finite.

To obtain a random variable $x = r\cos \theta \in (-\pi,\pi]$,
it is necessary that $r \in [0,\pi]$ and $\theta \in (-\pi,\pi)$.

Since $F^{-1}(0)=0$ and $F^{-1}(1)=\pi$,
a completely uniform variable $x$ distributed according to Equation \eqref{eq:gauss_proposal} is given by:
\begin{equation}\label{eq:gauss_angle}\begin{gathered}
    x = \sqrt{-\frac{2}{k}\log\left[1-y_1\left(1-e^{-\frac{k}{2}\pi^2}\right)\right]}\cos\left[2\pi\left(y_2-\frac{1}{2}\right)\right] \\
    \mathrm{with}\ \begin{dcases}
                         y_1 \in [0,1] \\
                         y_2 \in (0,1)
                   \end{dcases}
\end{gathered}\end{equation}

\subsection*{Local Metropolis-Hastings}
Now that the proposal distribution $\mathcal P(x_\mathrm{old}\to x_\mathrm{new})$ has been chosen,
the local algorithm can be defined as follows:
\begin{itemize}
    \item Select one link $\mathcal L$.
    \item Identify $\mathcal S_1$ and $\mathcal S_2$, the staples connected to $\mathcal L$ (Figure \ref{fig:staples}).
    \item Compute $W$, and then $k$ and $x_\mathrm{old}$:
        \[\begin{aligned}
            &W = U[\mathcal S_1] + U[\mathcal S_2],
            &
            &\begin{dcases}
                k = \beta |W| \\
                x_\mathrm{old} = \arg\left(WU[\mathcal L]\right)
            \end{dcases}
        \end{aligned}\]
    \item Extract $x_\mathrm{new}$ from $\mathcal P(x_\mathrm{old}\to x_\mathrm{new})$ using the algorithm of Equation \eqref{eq:gauss_angle}.
    \item Accept the proposal with probability (Equations (\ref{eq:acceptance}, \ref{eq:local_pdf}, \ref{eq:gauss_proposal})):
        \[\begin{aligned}
            \mathcal A(x_\mathrm{old}\to x_\mathrm{new})
                 &=  \min\left\{\frac{p(x_\mathrm{new};u_\mathrm{fix})\mathcal P(x_\mathrm{new}\to x_\mathrm{old})}
                                     {p(x_\mathrm{old};u_\mathrm{fix})\mathcal P(x_\mathrm{old}\to x_\mathrm{new})},1\right\} \\[.5em]
                 &= \min\left\{e^{k\left(\cos(x_\mathrm{new})-\frac{1}{2}x_\mathrm{old}^2-\cos(x_\mathrm{old})+\frac{1}{2}x_\mathrm{new}^2\right)},1\right\}
        \end{aligned}\]
        That is, generate a random variable $p$ from the uniform distribution in $[0,1)$,
        and accept the proposal if:
        \[
            p < e^{k\left(\cos(x_\mathrm{new})-\frac{1}{2}x_\mathrm{old}^2-\cos(x_\mathrm{old})+\frac{1}{2}x_\mathrm{new}^2\right)}
        \]
    \item If the proposal is accepted, set the corresponding new value for the link variable $U[\mathcal L]$
        \[
            U[\mathcal L] \mapsfrom e^{i(x_\mathrm{new}-\arg W)}
        \]
\end{itemize}

The local algorithm should then iterate over all links in the lattice to be ergodic, as discussed in Chapter \ref{ch:mc}.
Every sweep over all lattice links will be considered a step of the Markov chain,
and the expectation values will be evaluated on this sequence.

\subsection*{Local algorithm implementation}
Here is provided a C++17 implementation of the local algorithm using the {\ttfamily Lattice} class introduced in Chapter \ref{ch:lattice}.

It is necessary to define first a way to sample the proposal distribution of Equation \eqref{eq:gauss_proposal} and the uniform distribution.

In the C++ Standard Library, probability distributions are defined in terms of a generic templated type {\ttfamily URNG},
and the same convention will be used here.

The uniform distribution is already present in the Standard, implemented as the {\ttfamily uniform\_real\_distribution} class,
and, with its default parameters, corresponds to the uniform distribution in the interval $[0,1)$.
It will be aliased as follows:
\begin{lstlisting}[caption={Uniform distribution sampling function}]
using UniformDouble = uniform_real_distribution<double>;
\end{lstlisting}

The algorithm of Equation \eqref{eq:gauss_angle} can be implemented using the interface of Standard's URNGs:
\begin{lstlisting}[caption={Gaussian angle distribution sampling function}]
template <class URNG>
double gauss_angle(double k, URNG &rng)
{   
    // y1 unif in [0,1], y2 unif in (0,1)
    double y1 = (double)(rng()-rng.min())/(rng.max()-rng.min());
    double y2 = (double)(rng()-rng.min()+1lu)
                         / (rng.max()-rng.min()+2lu);
    
    double r = sqrt(-2./k*log(1.-y1*(1.-exp(-0.5*k*pi*pi))));
    double theta = 2.*pi*(y2-0.5);
    
    return r*cos(theta);
}
\end{lstlisting}

The local algorithm makes use of staples objects.
The same paradigm used in Chapter \ref{ch:lattice} is employed here to define the {\ttfamily Staple} type.
A staple can be identified given its connected link and the direction of the staple's first link.
For example, the staples $\mathcal S_1$ and $\mathcal S_2$ of Figure \ref{fig:staples} can be identified respectively with
$(\mathcal L, -1)$ and $(\mathcal L, 1)$.
The two connected staples employed in the local algorithm can then be obtained from the link $\mathcal L$ and its two ortogonal directions.

All this staples interface is implemented in the following way:
\begin{lstlisting}[caption={Staple type}]
using Staple = array<Link,3>;

// nu is the first staple link direction
Staple conn_staple(Link link, int nu)
{
    auto [s,mu] = link;
    return Staple{Link{s+hat(mu),nu},
                  Link{s+hat(mu)+hat(nu),-mu},
                  Link{s+hat(nu),-nu}};
}

// The order of returned staples is not considered
array<Staple,2> conn_staples(Link link)
{
    int nu;
    // Select ortogonal directions
    switch (link.mu) {
        case 1:; case -1: nu = 2; break;
        case 2:; case -2: nu = 1; break;
        default: throw runtime_error("Invalid mu");
    }
    return array<Staple,2>{conn_staple(link,nu),
                           conn_staple(link,-nu)};
}
\end{lstlisting}

Before updating links, it is recommended to start from a \emph{hot} configuration,
\ie with all link variables sampled uniformly in $U(1)$.
This starting configuration usually leads to shorter thermalization times.
\begin{lstlisting}[caption={Hot configuration}]
template <class URNG>
void set_hot(Lattice &lat, URNG &rng)
{
    for (int mu : {1,2}) {
        for (Site s : lat.sites()) {
            double theta = UniformDouble(0.,2.*pi)(rng);
            lat.set_link(Link{s,mu},exp(1i*theta));
        }
    }
}
\end{lstlisting}

The local algorithm is finally:
\begin{lstlisting}[caption={Local algorithm}]
// Return 1.0 if move is accepted, 0.0 if not
template <class URNG>
double local_update(Lattice &lat, Link link, URNG &rng)
{
    auto [S_1,S_2] = conn_staples(link);
    
    cmplx W = lat.s_line(S_1) + lat.s_line(S_2);
    double k = lat.beta()*abs(W);
    double x_old = arg(W*lat.s_line(link));
    
    double x_new = gauss_angle(k,rng);
    
    double p = exp(k*(cos(x_new)+pow(x_new,2)/2.
                     -cos(x_old)-pow(x_old,2)/2.));
    
    if (UniformDouble()(rng)<p) {
        cmplx u_new = exp(1i*(x_new-arg(W)));
        lat.set_link(link,u_new);
        return 1.;
    }
    else return 0.;
}
\end{lstlisting}

The local algorithm is then iterated over each lattice link:
\begin{lstlisting}[caption={Local sweep}]
template <class URNG>
double local_sweep(Lattice &lat, URNG &rng)
{
    double accept = 0.;
    for (int mu : {1,2}) {
        for (Site s : lat.sites()) {
            accept += local_update(lat,Link{s,mu},rng);
        }
    }
    return accept/pow(lat.N(),2)/2.;
}
\end{lstlisting}
A local sweep will be considered as an iteration of the Markov chain.

\section{First results}

The line of fixed constant volume $\beta/N^2=1/80$ is chosen with the following parameters:

\begin{table}[!htb]
    \centering
    \input{tables/local_cont.tex}
    \caption{Parameters considered for the local algorithm}
    \label{tab:local_cont}
\end{table}

\subsection*{Metropolis-Hastings acceptance}

As expected, the Metropolis-Hastings acceptance converges to 1 when approaching the continuum limit (Figure \ref{fig:local_acc}),
since, if $\beta$ increases, the $k\propto\beta$ parameter of Figure \ref{fig:gauss} increases, and the Laplace's method becomes more accurate.

\begin{figure}[!htb]
    \centering
    \import{gfx/}{local_acc.pgf}
    \caption{Local algorithm acceptance}
    \label{fig:local_acc}
\end{figure}

\subsection*{Plaquette mean energy}
For a $24\times24$ lattice with $\beta=7.2$, it is reported in \cite{durr-hoelbling:2005} the following value for the plaquette mean energy:
\begin{equation}\label{eq:plaq_durr}
    E_{\mathcal P} = 0.52040(39) \quad \mathrm{(Durr-Hoelbling)}
\end{equation}

Running the local algorithm for $10^6$ iterations, and discarding the first $20\%$ of the data as thermalization, the local algorithm obtained:
\begin{equation}\label{eq:plaq_local}
    E_{\mathcal P} = \input{tables/plaq_local.tex} \quad \mathrm{(Local\ algorithm)}
\end{equation}
which is compatible with \eqref{eq:plaq_durr}.

However, problems arise in the attempt to evaluate the continuum limit of the plaquette mean energy (Figure \ref{fig:local_cont_energy}).
The values of the two points on the left seem to be erroneous even if the plot of the plaquette energies history
does not show any evident problem (Figure \ref{fig:local_energy_history}).

\begin{figure}[!htb]
    \centering
    \import{gfx/}{local_cont_energy.pgf}
    \caption{Biased plaquette energy continuum limit obtained with the local algorithm}
    \label{fig:local_cont_energy}
\end{figure}

\begin{figure}[!htb]
    \centering
    \import{gfx/}{local_energy_history.pgf}
    \caption{Plaquette energy history obtained with the local algorithm}
    \label{fig:local_energy_history}
\end{figure}

The reason of these biased measures can be seen examining the behaviour of the topological charge.

\subsection*{Topological freezing}

Plotting the histories of the total topological charge values, the problem can be identified.
The second last plot of Figure \ref{fig:local_charge_history} show that the topological charge remains at the same value for a very long number of iterations,
while the last plot shows a complete charge freezing.

\begin{figure}[!htb]
    \centering
    \import{gfx/}{local_charge_history.pgf}
    \caption{Topological charge history obtained with the local algorithm}
    \label{fig:local_charge_history}
\end{figure}

The critical behaviour of the integrated autocorrelation time can be seen in Figure \ref{fig:local_charge_corr}.
The last two points are excluded from this plot because of the insufficient number of independent configurations.

\begin{figure}[!htb]
    \centering
    \import{gfx/}{local_charge_corr.pgf}
    \caption{Divergent integrated autocorrelation time of the topological charge obtained with the local algorithm}
    \label{fig:local_charge_corr}
\end{figure}

The topological susceptibility $\chi$ is proportional to the second moment of the topological charge:
\[
    \frac{\chi}{g^2} = \frac{\left<Q^2\right>}{a^2g^2N^2} = \frac{\left<Q^2\right>}{N^2}\beta
\]
Thus, the extrapolation of the continuum limit is very problematic in the presence of topological freezing (Figure \ref{fig:local_cont_susc}).

\begin{figure}[!htb]
    \centering
    \import{gfx/}{local_cont_susc.pgf}
    \caption{Biased topological susceptibility continuum limit obtained with the local algorithm}
    \label{fig:local_cont_susc}
\end{figure}

\section{Charge clusters}

\subsection*{Charge colormaps}

To get an insight of what happens in topological freezing regime,
it can be useful to find a method to visualize the spatial distribution of the local topological charge,
which was defined as:
\[
    q(s) \equiv \frac{1}{2\pi}\arg U[\mathcal P_{12}(s)]
\]
This means that its value can be in $(-1/2,1/2]$,
and every plaquette $\mathcal P_{12}(s)$ can either give o positive or negative contibution to the total charge.
The most appropriate way to represent this scenario, is with a diverging colormap, that is,
a colormap made of two contrasting colors with varying lightness and saturation that meet in the middle at an unsaturated color.
Each color represents positive or negative numbers,
with values of lightness and saturation that are perceived by the viewer as monotonically increasing.
In this way, distances between numbers can be visualized as distances between colors.

The diverging colormap chosen to represent the local charges is the \emph{Colobrewer - RdBu} map \cite{colorbrewer}.
To increase the range of visible values, the charges are first mapped using a logarithmic scale.
The typical values of the total charge are of the unit order, and it is the result of the summation of all the $N^2$ local contributions.
For this reason, the scale chosen to visualize the data ranges from $1/N^2$ to $100/N^2$, both in the positive and in the negative directions.
The values in $(-1/N^2,1/N^2)$ are mapped to the central white \emph{zero}. Values that are out of the scale have the same color as extreme values.

However, plotting the colormap of the local charges without preprocessing does not produce useful results (Figure \ref{fig:raw_charge_cmap}).
The values are very erratic, and an internal structure cannot be identified.

\begin{figure}[!htb]
	\centering
    \import{gfx/}{raw_charge_cmap.pgf}
    \caption{Local charge colormap without preprocessing}
    \label{fig:raw_charge_cmap}
\end{figure}

\subsection*{Charge smoothing}

What causes such a behaviour are ultraviolet fluctuations of the lattice local charge on wavelengths of the order of the lattice spacing $a$ \cite{teper:1985},
and they overshadow the typical fluctuations of the continuous charge on a certain wavelength $\rho$. 
This problem can be overcome approaching the continuum limit. Indeed, the lattice spacing becomes $a\ll\rho$,
and the unwanted ultraviolet fluctuations can be erased locally smoothing the fields over distances $\gg a$ but $\ll \rho$.

There are several methods to smooth link variables in Lattice QCD \cite{alexandrou:2017}.
However, in this toy model, the lattice topological charge is already well defined and integer valued.
For this reason, it is sufficient to smooth the local values of the charge to remove ultraviolet fluctuations,
which is easier, since they are simply matrices of values associated with lattice plaquettes.

These matrices can be represented as images, as discussed before,
and this parallelism suggests that the same techniques used to smooth colors in images can be employed to smooth local charge values.

For instance, in image processing, the Gaussian blur is a smoothing algorithm commonly used to remove image noise and for edge detection.
It consists in spreading the color of every pixel to all its neighbouring pixels, using a Gaussian weight, centered in the considered pixel.
The algorithm leaves the choice of the Gaussian $\sigma$ and a cutoff distance, beyond which the color is not spread.
In all following applications, the cutoff distance is set to $4\sigma$.

It is also needed to specify how the algorithm should handle pixels near the edges of the image.
In image processing, this is usually done treating pixels out of the bounds as if they were equal to those on the border.
For the smoothing of the topological charge, the lattice boundary conditions have to be reproduced,
and hence, periodic boundary conditions are implemented.

An example application of the Gussian blur is reported in Figure \ref{fig:cappadocia}, obtained used Scikit-image's Gaussian filter.
It can be seen how the smoothing procedure cancels the local squared structure,
and only the global colored spiral structure is visible with the higher value of $\sigma$.
Furthermore, the spiral is more evident in the blurred images, since the local structure overshadowes it.

\begin{figure}[!htb]
    \centering
    \import{gfx/}{cappadocia.pgf}
    \caption{\emph{Cappadocia Balloon Inflating Wikimedia Commons}, by Benh LIEU SONG, CC BY-SA 3.0}
    \label{fig:cappadocia}
\end{figure}

The parallelism with the ultraviolet fluctuations of the topological charge is evident.
Instead of the three components of the RGB colors, the local charge of a plaquette is spread.

The result of the Gaussian blur applied to the topological charge is reported in Figure \ref{fig:charge_blur}.
With a sufficiently high value of $\sigma$, ultaviolet fluctuations are removed,
and the cluster structure becomes evident.
If $\sigma$ is too high, also the topological charge fluctuations will cancel each others,
and too much information will be lost.

\begin{figure}[!htb]
    \centering
    \import{gfx/}{charge_blur.pgf}
    \caption{Topological charge Gaussian blur}
    \label{fig:charge_blur}
\end{figure}

The optimal values of sigma for the considered lattice spacing have been identified to be between $\sigma=1.5$ and $\sigma=2$.
And this still holds for bigger lattices with the same physical volume.

\subsection*{Topological freezing}
%Now that a method to visualize charge clusters is defined,
%it is possible to get an insight of what is happening in topological freezing regime
Now that a method to visualize charge clusters is defined,
it can be applied to get an insight of what is happening in topological freezing regime.

The history plots of Figure \ref{fig:local_charge_history} show that the fixed physical volume line of Table \ref{tab:local_cont}
heavily suffers from freezing if $N>24$.
Let then consider completely frozen configurations with $N=36,\ \beta=16.2$.

In Figure \ref{fig:freezing}, the blurred charge of four consecutive configurations are reported.
Charge clusters may move in the space or may be deformed,
but their contribution to the total charge remains the same.

\begin{figure}[!htb]
    \centering
    \import{gfx/}{freezing.pgf}
    \caption{Blurred charge maps after four consecutive local sweep updates}
    \label{fig:freezing}
\end{figure}

\subsection*{Charge tunneling}
If a solution to the topological freezing has to be found,
it is instructive to see what happens when charge tunneling is still achievable with a local algorithm.
Let then consider configurations with $N=24,\ \beta=7.2$,
in which the charge correlation time is already becomome high (Figure \ref{fig:local_charge_corr}) due to topological freezing,
but charge tunneling is still not so rare to prevent from collecting independent charge configurations.

The blurred charge maps after four consecutive sweeps are reported in Figure \ref{fig:local_inv},
and a charge tunneling has happened between the third and the fourth image.
A new cluster of negative red values has risen in a region in which a positive blue cluster was present,
and the topological charge moves from $Q=0$ to $Q=-1$.

\begin{figure}[!htb]
    \centering
    \import{gfx/}{local_inv.pgf}
    \caption{Charge tunneling with a local algorithm}
    \label{fig:local_inv}
\end{figure}

The idea of the cluster algorithm is then to create a Metropolis move that emulates this \emph{spontaneous} charge tunneling,
and this is the aim of the next Chapter.

%*****************************************
%*****************************************
%*****************************************
%*****************************************
%*****************************************

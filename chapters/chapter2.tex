%*****************************************
\chapter{Lattice Class Implementation}\label{ch:lattice}
%*****************************************

Nowadays, the availability of data and processing power is boosting more and more scientific research,
and it is driving a shift to computational and data-driven modes of discovery \cite{nielsen:2011}.

Computational scientific literature is, however, still suffering the lack of reproducibility \cite{reproducible:2009}.
It is difficult to verify most of the results presented at conferences and papers.

The sharing of data and code for verification purposes \cite{reproducible:2012}
is widely recognized as the most urgent practice that has to be adopted.

The implementation code in computational sciences has an importance that is comparable to the proof in deductive sciences \cite{topten}
and the technical descriptions of experiments in empirical sciences.

It is reported that the main obstacle that prevents code publication is the time it takes to clean up and document the work to prepare it for release and reuse \cite{obstacles}.
The best way to reduce the amount of this work is to use modern programming language techniques to make the code self-explanatory
and the closest possible to the problem that has to be solved \cite{best_pratices}.

Procedural programming, data abstraction, object-oriented and generic programming provide methods to express ideas directly in code:
it is possible to write algorithms in terms of abstract concepts hiding implementation details \cite{stroustrup:2013}.

Since Monte Carlo simulations of lattice gauge theories are, in general, very computationally expensive,
they need an efficient use of computer resources, and, therefore, they are implemented with programming languages that allow low level optimizations.

The most common used languages in this field are then FORTRAN, C and C++.
However, among them, the one that has the most comprehensive and flexible support of modern language features is C++,
especially C++11 and later standards.
It also has the largest standard library,
which allows to express algorithms in terms of well known and optimized objects, 
reducing also the dependency on external libraries.

In this work, the C++17 standard was adopted,
and everything has been implemented in terms of standard library constructs.

In this and later Chapters, the exposure and description of data structures and algorithms will be accompained with an implementation in C++17.
Particular effort was put into the design of the program in order to achieve an interface that reflects the notation of Chapter \ref{ch:toy_model}.
The algorithms of Chapters \ref{ch:local} and \ref{ch:cluster} will then be described with such an interface,
with the intent of expressing them in terms of abstract concepts, which are closer to the mathematical description of the system.
Such a paradigm makes the program to be shorter, easier to understand and to debug, and ensures that the lower level implementations are isolated and hidden from the rest of the code.
In this way, possible causes of errors and useless repetitions are also avoided.

Furthermore, the design of C++17 enables to define such an higher level interface without compromising the performance,
letting the compiler to perform lower level optimizations, rather than the programmer.
%Lower level implementation details will be kept hidden with the encapsulation methods the language provides,
%and they will be shown in Appendix \ref{ap:code}.

The aim of this Chapter is to implement the skeleton of the lattice theory.
It will be done in the form of a lattice class, which will have methods to access its basic objects,
compute and modify its Schwinger lines, and evaluate operators in terms of them.

%The interface the lattice class provides will then be used in Chapters \ref{ch:local} and \ref{ch:cluster} to define the Markov chain algorithms.

\section{Lattice objects}

The first things to define are the most basic data structures, \ie lattice objects,
and the first of them has to be a lattice site and its coordinates:
\begin{lstlisting}[caption={Site type}]
struct Site
{
    int x1;
    int x2;

    Site operator+(Site s) {
        return Site{x1+s.x1,x2+s.x2};
    }
};

Site hat(int mu)
{
    switch (mu) {
        case 1:; return Site{1,0};
        case -1: return Site{-1,0};
        case 2: return Site{0,1};
        case -2: return Site{0,-1};
        default: throw runtime_error("invalid mu");
    }
}
\end{lstlisting}
A site $s$ of coordinates $(x_1,x_2)$ is then {\ttfamily s = Site\{x1,x2\}}, and {\ttfamily s+hat(mu)} is the equivalent of $s+\hat\mu$.

A lattice link can be identified with a site and a direction $\texttt{mu} \in \{1,2,-1,-2\}$:
\begin{lstlisting}[caption={Link type}]
struct Link
{
    Site s;
    int mu;
};
\end{lstlisting}
The link $\mathcal L_\mu(x_1,x_2)$ will then be {\ttfamily Link\{Site\{x1,x2\},mu\}}.

%A generic lattice path is a sequence of links.
%The standard library {\ttfamily vector} is an indefinite sized container of object of the same type,
%and it already has methods to iterate over them.
%The implementation of a generic lattice path will then be just an alias for a vector of links:
%\begin{lstlisting}[caption={Path type}]
%using Path = vector<Link>;
%\end{lstlisting}

Lattice paths are sequences of links, and they can be implemented using Standard Library containers, which have built-it methods to iterate over them.
A plaquette is a four sized sequence of links, and hence the Standard Library {\ttfamily array<Link,4>} is suitable to implement it.
The size is specified at compile time,
and it enables more advanced compiler optimizations.

To define the analogous version of the plaquette $\mathcal P_{12}(s)$ of Figure \ref{fig:plaq}, the implementation is straightforward:
\begin{lstlisting}[caption={Plaquette type}]
using Plaq = array<Link,4>;

Plaq plaq_12(Site s)
{
    return Plaq{Link{s,1},
                Link{s+hat(1),2},
                Link{s+Site{1,1},-1},
                Link{s+hat(2),-2}};
}
\end{lstlisting}

\section{Lattice class declaration}

A $N \times N$ squared lattice with a given action parameter $\beta$ corresponds to the following {\ttfamily Lattice} class declaration.
The implementation will be shown in subsequent sections.

\begin{lstlisting}[caption={Lattice class}]
using cmplx = complex<double>;

class Lattice
{
    public:
        /* Lattice is initialized with its square side size N
           and the value of the action parameter beta */
        Lattice(int N, double beta);
        int N() const {return N_;}
        double beta() const {return beta_;}
        
        /* Vector of all lattice sites.
           Useful for iterating over them: */
        const vector<Site> &sites() const {return sites_;}
        
        // Observables: 
        double energy() const;
        double charge(Site) const;
        double total_charge() const;
        
        /* All the following methods are aware of
           boundary conditions and treat appropriately
           values out of bounds */
        
        // Compute Schwinger line over a generic Path:
        template <class Path>
        cmplx s_line(Path) const;
        
        // Set link variable link to value u:
        void set_link(Link link, cmplx u);
        
        // Local gauge transformation G(s):
        void local_gauge(Site s, cmplx G);
        
    private:
        int N_;
        double beta_;
        vector<Site> sites_;

        /* Link variables data:
           only the argument of them is stored,
           this prevents them from deviating from U(1) */
        vector<double> link_vars; // Link variables data
        
        /* link_vars indexing:
           (boundaries are implemented here) */
        int idx(Link) const;
};
\end{lstlisting}

The following class initialization gives an initial value of links variables and constructs the sequence of lattice sites that will be used to iterate over them.

\begin{lstlisting}[caption={Lattice constructor}]
/* Links are initialized with the value 1,
   which corresponds to a zero argument */
Lattice::Lattice(int N, double beta) :
    N_{N}, beta_{beta}, link_vars(2*N*N,0.)
{
    for (int x2=0; x2<N; x2++) {
        for (int x1=0; x1<N; x1++) {
            sites_.push_back(Site{x1,x2});
        }
    }
}
\end{lstlisting}

\subsection*{Lattice observables}

The observables considered in this work are the mean plaquette energy:
\[
    E_{\mathcal P} \equiv \frac{1}{N^2}\sum_s \beta\left(1-\Re U[\mathcal P_{12}(s)]\right)
\]
the local topological charge $q(s)$:
\[
    q(s) \equiv \frac{1}{2\pi}\arg U[\mathcal P_{12}(s)]
\]
and the total topological charge $Q$:
\[
    Q \equiv \sum_s q(s)
\]

Given a generic path, the corresponding Schwinger line $U$ is evaluated with the generic templated method {\ttfamily s\_line(Path)}.
The three observables are then implemented as:
\begin{lstlisting}[caption={Lattice observables}]
double Lattice::energy() const
{
    double sum = 0.;
    for (Site s : sites())
    {
        sum += 1.-real(s_line(plaq_12(s)));
    }
    return beta()*sum/pow(N(),2);
}

using pi = acos(-1.);

double Lattice::charge(Site s) const
{
    return arg(s_line(plaq_12(s)))/2./pi;
}

double Lattice::total_charge() const
{
    double sum = 0.;
    for (Site s : sites())
    {
        sum += charge(s);
    }
    return sum;
}
\end{lstlisting}

\subsection*{Schwinger lines}

Here is reported a lower level implementation of the Schwinger lines interface.
It is designed such that the following requirements are satisfied:
\begin{itemize}
    \item The method {\ttfamily s\_line(Path)} has to return the $U(1)$ complex number corresponding to the Schwinger line evaluated over the generic sequence of links {\ttfamily Path}.
    \item The method {\ttfamily set\_link(Link link, cmplx u)} has to set the link variable {\ttfamily link} to the value of the $U(1)$ complex number {\ttfamily u}.
        The direction has to be taken into account so that a successive call of {\ttfamily s\_line(link)} would return the value {\ttfamily u}.
    \item Both this methods have to handle link coordinates that are out of bounds, applying periodic boundary conditions.
    \item The link variables stored in memory have to correspond exactly to $U(1)$ numbers to avoid problems caused by the finite representation of numbers.
        This is done storing in memory the value of their argument, and building the corresponding $U(1)$ complex number when is needed.
\end{itemize}

Link variables arguments are stored in the private {\ttfamily vector<double>} container {\ttfamily link\_vars}.
The map between link coordinates {\ttfamily link} and the corresponding index of {\ttfamily link\_vars} is provided by the private method {\ttfamily idx}:
\begin{lstlisting}[caption={Link indexing}]
/* Only link variables pointing to positive directions
   are stored, i.e. top and right links.
   Links in the opposite directions
   are computed in terms of them taking the complex
   conjugate inside s_line and set_link methods. */


/* Apply periodic boundary conditions and return
   the corresponding index of link_vars.
   If link points to a negative direction,
   the index of the opposite link is returned.
   Links with contiguous x1 are contiguous in memory.
   The successive axis is x2.
   Links pointing to direction x1 are first in the ordering,
   then the ones pointing to x2. */
int Lattice::idx(Link link) const
{
    auto [x1,x2] = link.s;
    
    /* axis 0: points to x1
       axis 1: points to x2 */
    int axis = 0;
    /* Select right axis.
       If link points to negative directions,
       select the opposite one */
    switch (link.mu) {
        case 1: break;
        case -1: x1--; break;
        case 2: axis = 1; break;
        case -2: x2--; axis = 1; break;
        default: throw runtime_error("Invalid mu");
    }

    // Apply periodic boundary to coord x:
    x1 = (x1%N()+N()) % N();
    x2 = (x2%N()+N()) % N();
    
    return x1 + x2*N() + axis*pow(N(),2);
}
\end{lstlisting}

Once defined {\ttfamily idx}, the implementation of {\ttfamily s\_line(Link)} is trivial:
\begin{lstlisting}[caption={Link Schwinger line}]
template <>
cmplx Lattice::s_line(Link link) const
{
    cmplx u = exp(1i*link_vars[idx(link)]);
    if (link.mu>0) return u;
    else return conj(u);
}
\end{lstlisting}

This method can be generalized to every standard container of links:
\begin{lstlisting}[caption={Generic path Schwinger line}]
template <class Path>
cmplx Lattice::s_line(Path path) const
{
    cmplx prod = 1.;
    for (Link link : path)
    {
        prod *= s_line(link);
    }
    return prod;
}
\end{lstlisting}

The link updating is also self-explainatory:
\begin{lstlisting}[caption={Link updating}]
void Lattice::set_link(Link link, cmplx u)
{
    if (link.mu < 0) u = conj(u);
    link_vars[idx(link)] = arg(u);
}
\end{lstlisting}

\subsection*{Local gauge transform}

Finally, the local gauge transform of Figure \ref{fig:local_gauge},
applied to the site {\ttfamily s} is implemented as:
\begin{lstlisting}[caption={Local gauge transform}]
void Lattice::local_gauge(Site s, cmplx G)
{
    for (int mu : {1,2,-1,-2}) {
        Link link = Link{s,mu};
        set_link(link, G*s_line(link));
    }
}
\end{lstlisting}

%*****************************************
%*****************************************
%*****************************************
%*****************************************
%*****************************************

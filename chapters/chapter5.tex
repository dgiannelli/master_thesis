%************************************************
\chapter{Cluster Algorithm}\label{ch:cluster}
%************************************************

\section{Charge clusters}

\subsection*{Smoothing}
To get an insight of what happens in topological freezing regime,
it can be useful to find a method to visualize the spatial distribution of the local topological charge,
which was defined as:
\[
    q(s) \equiv \frac{1}{2\pi}\argU[\mathcal P(s)]
\]
This means that its value can be in $(-0.5,0.5]$,
and every plaquette $\mathcal P(s)$ can either give o positive or negative contibution to the total charge.
The most appropriate way to represent this scenario, is with a diverging colormap, that is,
a colormap made of two contrasting colors with varying lightness and saturation that meet in the middle at an unsaturated color.
Each color represents positive or negative numbers,
with values of lightness and saturation that are perceived by the viewer as monotonically increasing.
In this way, distances between numbers can be visualized as distances between colors.

The diverging colormap chosen to represent the local charges is the \emph{Colobrewer - RdBu} map \cite{colorbrewer}.
        





\begin{figure}[!htb]
    \centering
    \import{gfx/}{cappadocia.pgf}
    \caption{\emph{Cappadocia Balloon Inflating Wikimedia Commons}, by Benh LIEU SONG, CC BY-SA 3.0}
    \label{fig:cappadocia}
\end{figure}

\begin{figure}[!htb]
    \centering
    \import{gfx/}{charge_blur.pgf}
    \caption{Topological charge Gaussian blur}
    \label{fig:charge_blur}
\end{figure}


\begin{figure}[!htb]
    \centering
    \import{gfx/}{freezing.pgf}
    \caption{Freezing}
    \label{fig:freezing}
\end{figure}

\section{Cluster inversion}

\begin{figure}[!htb]
    \centering
    \import{gfx/}{local_inv.pgf}
    \caption{Charge tunneling}
    \label{fig:local_inv}
\end{figure}

\subsection*{Inverting links}

\subsection*{Gauge transforms}

\subsection*{Cluster building}

\begin{figure}[!htb]
    \centering
    \import{gfx/}{cluster_inv.pgf}
    \caption{Cluster algorithm inversion}
    \label{fig:cluster_inv}
\end{figure}

\begin{figure}[!htb]
    \centering
    \import{gfx/}{freezing_overcoming.pgf}
    \caption{Topological freezing overcoming with the cluster algorithm}
    \label{fig:freezing_overcoming}
\end{figure}

\begin{lstlisting}[caption={Cluster class declaration}]
class Cluster
{
    public:
        template <class URNG>
        Cluster(int N, int side, URNG&);

        Link gate() const {return gate_;};
        const vector<Link> &path() const {return path_;};
        const Staple &estaple() const {return estaple_;};
        const Staple &istaple() const {return istaple_;};
        vector<Link> links() const;
    private:
        int side;
        Site corner;

        Link gate_;
        vector<Link> path_;
        Staple estaple_, istaple_;
};
\end{lstlisting}

\begin{lstlisting}[caption={Cluster class implementation}]
template <class URNG>
Cluster::Cluster(int N, int side, URNG &rng) :
    side{side}
{
    int ran_pos = UniformInt(0,N*N-1)(rng);
    int x1 = ran_pos/N; // unif in [0,N-1]
    int x2 = ran_pos%N; // unif in [0,N-1]
    corner = Site{x1,x2}; // lower left corner

    int ran_muoff = UniformInt(0,2*4*side-1)(rng);
    int mu = ran_muoff%2 + 1; // unif in [1,2]
    int offset = ran_muoff/2; // unif in [0,4*side-1]

    // Build path
    int nu;
    switch (mu) {
        case 1: nu = 2; break;
        case 2: nu = 1; break;
        default: throw runtime_error("Invalid mu");
    }

    Site s = corner;
    for (int rho : {mu,nu,-mu,-nu}) {
        for (int i=0; i<side; i++) {
            path_.push_back(Link{s,rho});
            s = s + hat(rho);
        }
    }

    // Rotate path and extract gate
    rotate(path_.begin(), path_.begin()+offset, path_.end());
    gate_ = path_[0]; path_.erase(path_.begin());
    
    // Identify staples
    int mu_e; // External direction
    switch (offset/side) {
        case 0: mu_e = -nu; break;
        case 1: mu_e = mu; break;
        case 2: mu_e = nu; break;
        case 3: mu_e = -mu; break;
        default: throw runtime_error("invalid mu");
    }
    estaple_ = conn_staple(gate_,mu_e);
    istaple_ = conn_staple(gate_,-mu_e);
}

vector<Link> Cluster::links() const
{
    vector<Link> vec;

    int x1, x2;
    x2 = corner.x2+1;
    for (; x2<corner.x2+side; x2++) {
        x1 = corner.x1;
        for (; x1<corner.x1+side; x1++) {
            vec.push_back(Link{Site{x1,x2},1});
        }
    }
    x2 = corner.x2;
    for (; x2<corner.x2+side; x2++) {
        x1 = corner.x1+1;
        for (; x1<corner.x1+side; x1++) {
            vec.push_back(Link{Site{x1,x2},2});
        }
    }
    return vec;
}
\end{lstlisting}

\subsection*{Metropolis inversion}

\begin{lstlisting}[caption={Metropolis cluster update}]
template <class URNG>
double naive_cluster_update(Lattice &lat, int side, URNG &rng)
{
    Cluster cluster(lat.N(),side,rng);

    auto path = cluster.path();
    auto link_it = path.rbegin();
    for (; link_it<path.rend(); link_it++) {
        Link link = *link_it;
        cmplx u = lat.s_line(link);
        lat.local_gauge(link.s, conj(u));
    }

    cmplx US_e = lat.s_line(cluster.estaple());
    cmplx u_old = lat.s_line(cluster.gate());
    cmplx u_new = conj(u_old);

    double p = exp(lat.beta()*real(US_e*(u_new-u_old)));

    if (UniformDouble()(rng)<p) {
        lat.set_link(cluster.gate(),u_new);
        for (Link link : cluster.links()) {
            cmplx u = lat.s_line(link);
            lat.set_link(link,conj(u));
        }
        return 1.;
    }
    else return 0.;
}
\end{lstlisting}

\subsection*{Metropolis-Hastings proposal}
\begin{lstlisting}[caption={Metropolis-Hastings cluster update}]
template <class URNG>
double cluster_update(Lattice &lat, int side, URNG &rng)
{
    Cluster cluster(lat.N(),side,rng);

    auto path = cluster.path();
    auto link_it = path.rbegin();
    for (; link_it<path.rend(); link_it++) {
        Link link = *link_it;
        cmplx u = lat.s_line(link);
        lat.local_gauge(link.s, conj(u));
    }
    
    cmplx US_e = lat.s_line(cluster.estaple());
    cmplx US_i = lat.s_line(cluster.istaple());
    cmplx W_new = US_e + conj(US_i);
    cmplx W_old = US_e + US_i;
    
    double k_old = lat.beta()*abs(W_old);
    double k_new = lat.beta()*abs(W_new);

    cmplx u_old = lat.s_line(cluster.gate());
    double x_old = arg(W_old*u_old);
    
    double x_new = gauss_angle(k_new,rng);
    

    double p = exp(k_new*(cos(x_new)+pow(x_new,2.)/2.)
                  -k_old*(cos(x_old)+pow(x_old,2.)/2.))
              *erf(pi*sqrt(k_new/2.))/erf(pi*sqrt(k_old/2.))
              *sqrt(k_old/k_new);
    
    if (UniformDouble()(rng)<p) {
        cmplx u_new = exp(1i*(x_new-arg(W_new)));
        lat.set_link(cluster.gate(),u_new);
        for (Link link : cluster.links()) {
            cmplx u = lat.s_line(link);
            lat.set_link(link,conj(u));
        }
        return 1.;
    }
    else return 0.;
}
\end{lstlisting}

\begin{figure}[!htb]
    \centering
    \import{gfx/}{cluster_acc_cont.pgf}
    \caption{Cluster algorithms acceptance contunuum limit comparison}
    \label{fig:cluster_acc_cont}
\end{figure}

\begin{figure}[!htb]
	\centering
    \import{gfx/}{cluster_side_acc.pgf}
    \caption{Cluster side acceptance comparison}
    \label{fig:cluster_side_acc}
\end{figure}

%*****************************************
%*****************************************
%*****************************************
%*****************************************
%*****************************************

%************************************************
\chapter{Cluster Algorithm}\label{ch:cluster}
%************************************************

\section{Charge clusters}

\subsection*{Charge colormaps}

To get an insight of what happens in topological freezing regime,
it can be useful to find a method to visualize the spatial distribution of the local topological charge,
which was defined as:
\[
    q(s) \equiv \frac{1}{2\pi}\arg U[\mathcal P_{12}(s)]
\]
This means that its value can be in $(-1/2,1/2]$,
and every plaquette $\mathcal P_{12}(s)$ can either give o positive or negative contibution to the total charge.
The most appropriate way to represent this scenario, is with a diverging colormap, that is,
a colormap made of two contrasting colors with varying lightness and saturation that meet in the middle at an unsaturated color.
Each color represents positive or negative numbers,
with values of lightness and saturation that are perceived by the viewer as monotonically increasing.
In this way, distances between numbers can be visualized as distances between colors.

The diverging colormap chosen to represent the local charges is the \emph{Colobrewer - RdBu} map \cite{colorbrewer}.
To increase the range of visible values, the charges are first mapped using a logarithmic scale.
The typical values of the total charge are of the unit order, and it is the result of the summation of all the $N^2$ local contributions.
For this reason, the scale chosen to visualize the data ranges from $1/N^2$ to $100/N^2$, both in the positive and in the negative directions.
The values in $(-1/N^2,1/N^2)$ are mapped to the central white \emph{zero}. Values that are out of the scale have the same color as extreme values.

However, plotting the colormap of the local charges without preprocessing does not produce useful results (Figure \ref{fig:raw_charge_cmap}).
The values are very erratic, and an internal structure cannot be identified.

\begin{figure}[!htb]
	\centering
    \import{gfx/}{raw_charge_cmap.pgf}
    \caption{Local charge colormap without preprocessing}
    \label{fig:raw_charge_cmap}
\end{figure}

\subsection*{Charge smoothing}

What causes such a behaviour are ultraviolet fluctuations of the lattice local charge on wavelengths of the order of the lattice spacing $a$ \cite{teper:1985},
and they overshadow the typical fluctuations of the continuous charge on a certain wavelength $\rho$. 
This problem can be overcome approaching the continuum limit. Indeed, the lattice spacing becomes $a\ll\rho$,
and the unwanted ultraviolet fluctuations can be erased locally smoothing the fields over distances $\gg a$ but $\ll \rho$.

There are several methods to smooth link variables in Lattice QCD \cite{alexandrou:2017}.
However, in this toy model, the lattice topological charge is already well defined and integer valued.
For this reason, it is sufficient to smooth the local values of the charge to remove ultraviolet fluctuations,
which is easier, since they are simply matrices of values associated with lattice plaquettes.

These matrices can be represented as images, as discussed before,
and this parallelism suggests that the same techniques used to smooth colors in images can be employed to smooth local charge values.

For instance, in image processing, the Gaussian blur is a smoothing algorithm commonly used to remove image noise and for edge detection.
It consists in spreading the color of every pixel to all its neighbouring pixels, using a Gaussian weight, centered in the considered pixel.
The algorithm leaves the choice of the Gaussian $\sigma$ and a cutoff distance, beyond which the color is not spread.
In all following applications, the cutoff distance is set to $4\sigma$.

It is also needed to specify how the algorithm should handle pixels near the edges of the image.
In image processing, this is usually done treating pixels out of the bounds as if they were equal to those on the border.
For the smoothing of the topological charge, the lattice boundary conditions have to be reproduced,
and hence, periodic boundary conditions are implemented.

An example application of the Gussian blur is reported in Figure \ref{fig:cappadocia}, obtained used Scikit-image's Gaussian filter.
It can be seen how the smoothing procedure cancels the local squared structure,
and only the global colored spiral structure is visible with the higher value of $\sigma$.
Furthermore, the spiral is more evident in the blurred images, since the local structure overshadowes it.

\begin{figure}[!htb]
    \centering
    \import{gfx/}{cappadocia.pgf}
    \caption{\emph{Cappadocia Balloon Inflating Wikimedia Commons}, by Benh LIEU SONG, CC BY-SA 3.0}
    \label{fig:cappadocia}
\end{figure}

The parallelism with the ultraviolet fluctuations of the topological charge is evident.
Instead of the three components of the RGB colors, the local charge of a plaquette is spread.

The result of the Gaussian blur applied to the topological charge is reported in Figure \ref{fig:charge_blur}.
With a sufficiently high value of $\sigma$, ultaviolet fluctuations are removed,
and the cluster structure becomes evident.
If $\sigma$ is too high, also the topological charge fluctuations will cancel each others,
and too much information will be lost.

\begin{figure}[!htb]
    \centering
    \import{gfx/}{charge_blur.pgf}
    \caption{Topological charge Gaussian blur}
    \label{fig:charge_blur}
\end{figure}

The optimal values of sigma for the considered lattice spacing have been identified to be between $\sigma=1.5$ and $\sigma=2$.
And this still holds for bigger lattices with the same physical volume.

\subsection*{Topological freezing}
%Now that a method to visualize charge clusters is defined,
%it is possible to get an insight of what is happening in topological freezing regime
Now that a method to visualize charge clusters is defined,
it can be applied to get an insight of what is happening in topological freezing regime.

The history plots of Figure \ref{fig:local_charge_history} show that the fixed physical volume line of Table \ref{tab:local_cont}
heavily suffers from freezing if $N>24$.
Let then consider completely frozen configurations with $N=36,\ \beta=16.2$.

In Figure \ref{fig:freezing}, the blurred charge of four consecutive configurations are reported.
Charge clusters may move in the space or may be deformed,
but their contribution to the total charge remains the same.

\begin{figure}[!htb]
    \centering
    \import{gfx/}{freezing.pgf}
    \caption{Blurred charge maps after four consecutive local sweep updates}
    \label{fig:freezing}
\end{figure}

\subsection*{Charge tunneling}
If a solution to the topological freezing has to be found,
it is instructive to see what happens when charge tunneling is still achievable with a local algorithm.
Let then consider configurations with $N=24,\ \beta=7.2$,
in which the charge correlation time is already becomome high (Figure \ref{fig:local_charge_corr}) due to topological freezing,
but charge tunneling is still not so rare to prevent from collecting independent charge configurations.

The blurred charge maps after four consecutive sweeps are reported in Figure \ref{fig:local_inv},
and a charge tunneling has happened between the third and the fourth image.
A new cluster of negative red values has risen in a region in which a positive blue cluster was present,
and the topological charge moves from $Q=0$ to $Q=-1$.

\begin{figure}[!htb]
    \centering
    \import{gfx/}{local_inv.pgf}
    \caption{Charge tunneling with a local algorithm}
    \label{fig:local_inv}
\end{figure}

The idea of the cluster algorithm is then to create a Metropolis move that emulates this \emph{spontaneous} charge tunneling,

\section{Reverse cluster algorithm}

\subsection*{Inverting links}

Transforming each link variable inside a plaquette to its complex conjugate
has the effect of reversing the value of its local topological charge (Equation \eqref{eq:lat_loc_top_charge}):
\[\begin{aligned}
    q(s) &= \frac{1}{2\pi}\arg U[\mathcal P_{12}(s)] \\
         &\mapsto \frac{1}{2\pi}\arg U^*[\mathcal P_{12}(s)] = -q(s)
\end{aligned}\]
It has the same effect of taking the plaquette in the opposite direction:
\[
    U[\mathcal P_{21}(s)] = U^*[\mathcal P_{12}(s)]
\]

The plaquette energy is instead invariant under links conjugation:
\[\begin{aligned}
    E(s) &= 1-\Re\,U[\mathcal P_{12}(s)] \\
         &\mapsto 1-\Re\,U^*[\mathcal P_{12}] = E(s)
\end{aligned}\]

In terms of Markov chain sampling and Metropolis algorithm (Equation \ref{eq:acceptance_sym}),
a move that produces zero energy variation is a move that has acceptance one.

Thus, taking the complex conjugate of all links inside a space region does not change the energy of all plaquettes of this region.
What actually changes, and prevents this block move to be an effective Metropolis update,
is the energy of all plaquettes connected to the border links of this region.
Indeed, only one link in them is conjugated, and the energy changes, in general, if this link is not real valued.
It is then very unlikely that the move is accepted if the energies of a high number of plaquettes change all together,
especially close to the continuum limit, when $\beta$ is high, as the system is close to its \emph{ground state}.

However, there is still another move that leaves the energy and the charge unchanged,
and it is the gauge transformation.

\subsection*{Gauge transformations}
Local gauge transformations of the type of Equation \eqref{eq:site_gauge} and Figure \ref{fig:local_gauge} are now considered.
Each plaquette Wilson loop remains unchanged after such a transformation,
and hence also the topological charge and the energy are invariant.
The goal is then to find an appropriate gauge transformation that minimizes the energy variation of the cluster inversion described before.

It was observed before that if link variables along the border are real valued,
the energy of the corresponding external plaquettes does not change after the cluster inversion.
A possible strategy to reduce the energy variation is to find a set of local gauge transformations that set border link variables to $1$,
and then perform the cluster inversion.

Considering the lattice region of Figure \ref{fig:cluster}, and the link path surrounding it,
let $U_1, U_2, \ldots$ be the Schwinger lines evaluated over links $\mathcal L_1, \mathcal L_2, \ldots$,
and $s_1, s_2, \ldots$ be the starting sites of them.
The last link variable of the path, $U_{12}$, can be set to $1$ with the gauge transform of Figure \ref{fig:local_gauge}
with $G=U^*_{12}$ applied on site $s_{12}$.
The only other link variable of the path that is modified by this transformation is the previous one: $U_{11}$.
Applying then another local gauge trasformation with $G=U^*_{11}$ on site $s_{11}$,
also the value of $U_{11}$ can be set to $1$, without modifying the other value already set to $1$.
This sequence can be iterated to set all values $U_1,U_2,\ldots=1$.
However, the same procedure cannot be applied to the last remaining path link, $\mathcal L$,
because it would modify the value $U_{12}$, already set to $1$ with the first transformation.

All these gauge transformations do not modify the energy of the system,
and then, they can be accepted with acceptance one.
They can be seen as a preparatory step of the cluster inversion.

After such a chain of gauge transfomations,
reversing all links in the surrounded region would modify only one external plaquette,
the one enclosing the link $\mathcal L$.

The acceptance of the entire move can be evaluated with Equation \eqref{eq:acceptance_sym}.
Since only the energy of one single plaquette is modified,
the increasing of energy cannot be too high, and hence, the acceptance cannot be too low.

\begin{figure}[!htb]
    \centering
    \begin{tikzpicture}[x=1cm,y=1cm]
        \draw[step=1, help lines, dashed, color=black!30] (0,0) grid (4.5,4.5);
        \draw[->,thick,black] (0,0) -- (4.5,0) node [right] {$x_1$};
        \draw[->,thick,black] (0,0) -- (0,4.5) node [above] {$x_2$};

        %internal links
        %\draw[step=1,color=RoyalBlue] (1,1) grid (3,3);

        %gate
        \draw[->-,very thick,black,dashed] (2,1) -- node [black,above] {$\mathcal L$} (3,1);

        %path
        \draw[->-,very thick,RoyalBlue] (3,1) -- node [black,below] {$\mathcal L_1$} (4,1);
        \draw[->-,very thick,RoyalBlue] (4,1) -- node [black,right] {$\mathcal L_2$} (4,2);
        \draw[->-,very thick,RoyalBlue] (4,2) -- node [black,right] {$\mathcal L_3$} (4,3);
        \draw[->-,very thick,RoyalBlue] (4,3) -- node [black,right] {$\mathcal L_4$} (4,4);
        \draw[->-,very thick,RoyalBlue] (4,4) -- node [black,above] {$\mathcal L_5$} (3,4);
        \draw[->-,very thick,RoyalBlue] (3,4) -- node [black,above] {$\mathcal L_6$} (2,4);
        \draw[->-,very thick,RoyalBlue] (2,4) -- node [black,above] {$\mathcal L_7$} (1,4);
        \draw[->-,very thick,RoyalBlue] (1,4) -- node [black,left] {$\mathcal L_8$} (1,3);
        \draw[->-,very thick,RoyalBlue] (1,3) -- node [black,left] {$\mathcal L_9$} (1,2);
        \draw[->-,very thick,RoyalBlue] (1,2) -- node [black,left] {$\mathcal L_{10}$} (1,1);
        \draw[->-,very thick,RoyalBlue] (1,1) -- node [black,below] {$\mathcal L_{11}$} (2,1);

        %internal staple
        \draw[->-,very thick,Maroon] (3,1) -- (3,2);
        \draw[->-,very thick,Maroon] (3,2) -- node [black,above] {$\mathcal S_i$} (2,2);
        \draw[->-,very thick,Maroon] (2,2) -- (2,1);

        %external staple
        \draw[->-,very thick,Maroon] (3,1) -- (3,0);
        \draw[->-,very thick,Maroon] (3,0) -- node [black,above] {$\mathcal S_e$} (2,0);
        \draw[->-,very thick,Maroon] (2,0) -- (2,1);
    \end{tikzpicture}
    \caption{Cluster of links with its surrounding path}
    \label{fig:cluster}
\end{figure}

\subsection*{Reverse cluster update}

Here are summarized the steps of the first implementation of a cluster algotithm.
This version will be referred as the \emph{reverse cluster update}.
The details about the construction of clusters of the kind of Figure \ref{fig:cluster}
will be explained in the next section.

\begin{itemize}
    \item
        Select a random cluster of the kind of Figure \ref{fig:cluster},
        identifying its surrounding path $\mathcal L_1, \mathcal L_2, \ldots$,
        its unfixed link $\mathcal L$, which will be reffered as the \emph{gate} link,
        and its internal and external staples $\mathcal S_i, \mathcal S_e$.
    \item 
        Starting from the last link of the path,
        $\mathcal L_{11}$ in the case of Figure \ref{fig:cluster},
        perform a gauge transformation of the kind of Figure \ref{fig:local_gauge}
        with $G=U^*_{11}$ applied on the starting site of the considered link,
        and iterate the procedure until reaching the first link of the path $\mathcal L_1$.
    \item
        The action variation due to the cluster inversion is:
        \[
            \Delta S = -\beta(\Re\,(U[\mathcal S_e](U^*[\mathcal L]-U[\mathcal L])))
        \]
        The update is then accepted with probability:
        \[
            \mathcal A = e^{-\Delta S}
        \]
    \item 
        If the update is accepted,
        set to its own conjugate the gate link and all internal cluster links.
\end{itemize}

The simulation should then alternate sweeps of local updates,
which are efficient in sampling the Boltzmann distribution at fixed topology,
and cluster updates, which are efficient in tunneling between topological sectors.

\subsection{Reverse cluster implementation}
The implementation follows the same pattern.
The cluster building interface is here declared without definition,
and it is used to define the reverse cluster update.

In the following discussion,
clusters will be considered random lattice squares with a given side,
and a randomly chosen gate link and path orientation.

Even though the algorithm can be extended to any generic simply connected cluster,
there are no reasons to believe that, in a squared lattice,
other generic shapes would achieve so much better performances than squares
to justify a more complicated and less efficient implementation of them.

This ansatz is justified by the shapes of clusters of Figure \ref{fig:freezing}.
They have a typical size determined by the physical volume of the system,
but they have very heterogeneous and variadic shapes. 
For this reason,
it is impossible to identify a more specialized shape that would fit the cluster structure.

It may be possible that using sequences of different shapes would be more efficient to achieve charge tunneling.
However, since performances of the algorithm obtained with squared clusters have already been considered very satisfactory,
such a detailed study of cluster shapes have not be considered of great interest.

The declaration of the squared {\ttfamily Cluster} class is outlined as follows:

\begin{lstlisting}[caption={Cluster class declaration}]
class Cluster
{
    public:
        template <class URNG>
        /* N <- lattice side
           side <- squared cluster side
           rng <- random number generator */
        Cluster(int N, int side, URNG &rng);

        /* Return respectively:
           the gate link,
           the surrounding path,
           the esternal staple,
           the internal staple
           and the internal links */
        Link gate() const {return gate_;};
        const vector<Link> &path() const {return path_;};
        const Staple &estaple() const {return estaple_;};
        const Staple &istaple() const {return istaple_;};
        vector<Link> links() const;
    private:
        int side;
        Site corner;

        Link gate_;
        vector<Link> path_;
        Staple estaple_, istaple_;
};
\end{lstlisting}

Using this {\ttfamily Cluster} class interface, the reverse cluster algorithm is the following:

\begin{lstlisting}[caption={Reverse cluster update}]
// Return 1.0 if move is accepted, 0.0 if not
template <class URNG>
double reverse_cluster(Lattice &lat, int side, URNG &rng)
{
    Cluster cluster(lat.N(),side,rng);

    auto path = cluster.path();

    /* rbegin and rend return a reverse iterator
       useful to iterate backward on path links */
    auto link_it = path.rbegin();
    for (; link_it<path.rend(); link_it++) {
        Link link = *link_it;
        cmplx u = lat.s_line(link);
        lat.local_gauge(link.s, conj(u));
    }

    cmplx US_e = lat.s_line(cluster.estaple());
    cmplx u_old = lat.s_line(cluster.gate());
    cmplx u_new = conj(u_old);

    double p = exp(lat.beta()*real(US_e*(u_new-u_old)));

    if (UniformDouble()(rng)<p) {
        lat.set_link(cluster.gate(),u_new);
        for (Link link : cluster.links()) {
            cmplx u = lat.s_line(link);
            lat.set_link(link,conj(u));
        }
        return 1.;
    }
    else return 0.;
}
\end{lstlisting}

\subsection*{Cluster building}

Here is now discussed the construction of a random squared cluster of the kind of Figure \ref{fig:cluster}.
The algorithm is made of the following steps:

\begin{itemize}
    \item
        Select randomly a lattice site, which will be the cluster lower left corner,
        the orientation of the surrounding path and the gate link.
    \item
        Build the squared path of size $\mathrm{side}\times 4$ starting from the lower left corner along the selected direction.
    \item
        Rotate the path so that the gate will be its first link,
        and remove the gate from the path.
    \item Identify the internal and the external staples connected to the gate.
    \item
        Starting from the corner, select all internal horizontal and vertical links inside the square.
\end{itemize}

This can be implemented as a {\ttfamily Cluster} class as follows:

\begin{lstlisting}[caption={Cluster class implementation}]
template <class URNG>
Cluster::Cluster(int N, int side, URNG &rng) :
    side{side}
{
    /* Select random corner,
       starting path direction
       and the path position of the gate */

    int ran_pos = UniformInt(0,N*N-1)(rng);
    int x1 = ran_pos/N; // unif in [0,N-1]
    int x2 = ran_pos%N; // unif in [0,N-1]
    corner = Site{x1,x2}; // lower left corner

    int ran_muoff = UniformInt(0,2*4*side-1)(rng);
    int mu = ran_muoff%2 + 1; // unif in [1,2]
    int offset = ran_muoff/2; // unif in [0,4*side-1]


    // Build path

    int nu;
    switch (mu) {
        case 1: nu = 2; break;
        case 2: nu = 1; break;
        default: throw runtime_error("Invalid mu");
    }

    Site s = corner;
    for (int rho : {mu,nu,-mu,-nu}) {
        for (int i=0; i<side; i++) {
            path_.push_back(Link{s,rho});
            s = s + hat(rho);
        }
    }

    // Rotate path and extract gate
    rotate(path_.begin(), path_.begin()+offset, path_.end());
    gate_ = path_[0]; path_.erase(path_.begin());
    
    // Identify staples
    int mu_e; // External direction
    switch (offset/side) {
        case 0: mu_e = -nu; break;
        case 1: mu_e = mu; break;
        case 2: mu_e = nu; break;
        case 3: mu_e = -mu; break;
        default: throw runtime_error("invalid mu");
    }
    estaple_ = conn_staple(gate_,mu_e);
    istaple_ = conn_staple(gate_,-mu_e);
}

/* Select internal links.
   This function is lazily evaluated
   only if the inversion move is accepted */
vector<Link> Cluster::links() const
{
    vector<Link> vec;

    int x1, x2;
    x2 = corner.x2+1;
    for (; x2<corner.x2+side; x2++) {
        x1 = corner.x1;
        for (; x1<corner.x1+side; x1++) {
            vec.push_back(Link{Site{x1,x2},1});
        }
    }
    x2 = corner.x2;
    for (; x2<corner.x2+side; x2++) {
        x1 = corner.x1+1;
        for (; x1<corner.x1+side; x1++) {
            vec.push_back(Link{Site{x1,x2},2});
        }
    }
    return vec;
}
\end{lstlisting}

\subsection*{Topological freezing overcoming}
Here the reverse cluster update is immediately put to the test,
showing if it actually overcomes the topological freezing.

In Figure \ref{fig:cluster_inv}, the cluster algorithm is applied to the central highlighted square.
The move is accepted, the total charge has changed.
The field configuration has been turned up,
and the blurred charge map shows that the selected cluster has inverted his color:
the red structure has turned to one structure of similar form,
and even the smaller blue twist enclosed inside it has reversed his color.

\begin{figure}[!htb]
    \centering
    \import{gfx/}{cluster_inv.pgf}
    \caption{Cluster algorithm inversion}
    \label{fig:cluster_inv}
\end{figure}

For illustrative purposes,
a preview of the simulation described in the next chapter is reported in Figure \ref{fig:freezing_overcoming},
in which are compared the histories of the topological charge obtained with the local and the cluster algorithm.
With this parameter choice, the local algorithm is freezed in the same topological sector for all the duration of the simulation,
while the cluster algorithm does not show any evidence of topological freezing.

\begin{figure}[!htb]
    \centering
    \import{gfx/}{freezing_overcoming.pgf}
    \caption{Topological freezing overcoming with the cluster algorithm}
    \label{fig:freezing_overcoming}
\end{figure}

\section{Gauss cluster algorithm}

There is still an improvement of the algorithm that can be made to increase the acceptance.
In the reverse cluster update,
the acceptance is expected to be reasonably high since only one plaquette energy is modified.
However, it is not necessary that with another choice for the gate variable the acceptance would be lower,
even if the plaquette energies modified are two instead of one.

Indeed, the choice can be made in a way that is similar to the one used in Chapter \ref{ch:local}.
The idea is to choose a value for the gate variable using the Gaussian distribution,
using the reversed staple value to find the parameters $x$ and $k$.
In this way, the new gate value is more likely to be at thermal equilibrium configuration with the two staples.

However, the acceptance will not be as high as for the local update,
since all the other cluster link variables are chosen arbitrarily,
and not sampled from their heat-bath distribution, which is unknown.

Unlike the case of the local algorithm,
the proposal distributions $\mathcal P(x_\mathrm{old}\to x_\mathrm{new})$ and $\mathcal P(x_\mathrm{new}\to x_\mathrm{old})$
are not the same, since the $k$ parameter,
which depends on the staples Schwinger line, is different before and after the move.
For this reason, the normalization factor of the Gaussian cannot be ignored:
\[\begin{aligned}
    \int_{-\pi}^\pi\mathrm dx\,e^{-kx^2/2} &= 2\int_0^\pi\mathrm dx\,e^{-kx^2/2} \\
                                           &= 2\sqrt{\frac{2}{k}}\int_0^{k\pi^2/2}\mathrm dx\,e^{-x^2} = \sqrt{\frac{2\pi}{k}}\mathrm{erf}\left(\frac{k}{2}\pi^2\right)
\end{aligned}\]
where $\mathrm{erf}(x) \equiv \frac{2}{\sqrt\pi}\int_0^x\mathrm dt\,e^{-t^2}$ is the error function.
The proposal distribution is then:
\begin{equation}\label{eq:gauss_cluster_proposal}
    \mathcal P(x_\mathrm{old}\to x_\mathrm{new})\sqrt{\frac{k_\mathrm{new}}{2\pi}}\frac{e^{-k_\mathrm{new}x^2/2}}{\mathrm{erf}\left(\frac{k_\mathrm{new}}{2}\pi^2\right)}
\end{equation}

\subsection*{Gauss cluster update}

The Gauss cluster update algorithm can now be defined:

\begin{itemize}
    \item
        Create a cluster and perform the chain of local gauge transformations in the same way as in the reverse cluster update.
    \item
        Compute the sum of connected staples Schwinger lines before and after the cluster inversion:
        \[\begin{aligned}
            W_\mathrm{old} &\equiv U[\mathcal S_\mathrm{e}] + U[\mathcal S_\mathrm{i}] &
            W_\mathrm{new} &\equiv U[\mathcal S_\mathrm{e}] + U^*[\mathcal S_\mathrm{i}]
        \end{aligned}\]
    \item
        Compute $k_\mathrm{old}, k_\mathrm{new}, x_\mathrm{old}$ using the following definitions:
        \[\begin{gathered}
            \begin{aligned}
                k_\mathrm{old} &\equiv \beta|W_\mathrm{old}| & k_\mathrm{new} &\equiv \beta|W_\mathrm{new}|
            \end{aligned}\\
            x_\mathrm{old} \equiv \arg(W_\mathrm{old}U[\mathcal L])
        \end{gathered}\]
        where $\mathcal L$ is the gate link.
    \item
        Sample $x_\mathrm{new}$ from $\mathcal P(x_\mathrm{old}\to x_\mathrm{new})$ of Equation \eqref{eq:gauss_cluster_proposal}
        using the algorithm of Equation \eqref{eq:gauss_angle}.
    \item
        Accept the proposal with probability:
        \[\begin{aligned}
            \mathcal A(x_\mathrm{old}\to x_\mathrm{new}) &= e^{-\Delta S}\frac{\mathcal P(x_\mathrm{new}\to x_\mathrm{old})}{\mathcal P(x_\mathrm{old}\to x_\mathrm{new})} \\
%            = e^{k_\mathrm{new}(\cos x_\mathrm{new}+x_\mathrm{new}^2/2)-k_\mathrm{old}(\cos x_\mathrm{old}+x_\mathrm{old}^2/2)}
%              \sqrt{\frac{k_\mathrm{old}}{k_\mathrm{new}}}\frac{\mathrm{erf}\left(\pi\sqrt{\frac{k_\mathrm{new}}{2}}\right)}{\mathrm{erf}\left(\pi\sqrt{\frac{k_\mathrm{old}}{2}}\right)}\\
            &= \frac{e^{k_\mathrm{new}(\cos x_\mathrm{new}+x_\mathrm{new}^2/2)}}{e^{k_\mathrm{old}(\cos x_\mathrm{old}+x_\mathrm{old}^2/2)}}
              \sqrt{\frac{k_\mathrm{old}}{k_\mathrm{new}}}\frac{\mathrm{erf}\left(\pi\sqrt{\frac{k_\mathrm{new}}{2}}\right)}{\mathrm{erf}\left(\pi\sqrt{\frac{k_\mathrm{old}}{2}}\right)}
        \end{aligned}\]
    \item
        If the proposal is accepted, update the gate link as follows:
        \[
            U[\mathcal L] \mapsfrom e^{i(x_\mathrm{new}-\arg W_\mathrm{new})}
        \]
        and set all internal cluster links to its own conjugate.
\end{itemize}

\begin{lstlisting}[caption={Metropolis-Hastings cluster update}]
template <class URNG>
double cluster_update(Lattice &lat, int side, URNG &rng)
{
    Cluster cluster(lat.N(),side,rng);

    auto path = cluster.path();
    auto link_it = path.rbegin();
    for (; link_it<path.rend(); link_it++) {
        Link link = *link_it;
        cmplx u = lat.s_line(link);
        lat.local_gauge(link.s, conj(u));
    }
    
    cmplx US_e = lat.s_line(cluster.estaple());
    cmplx US_i = lat.s_line(cluster.istaple());
    cmplx W_new = US_e + conj(US_i);
    cmplx W_old = US_e + US_i;
    
    double k_old = lat.beta()*abs(W_old);
    double k_new = lat.beta()*abs(W_new);

    cmplx u_old = lat.s_line(cluster.gate());
    double x_old = arg(W_old*u_old);
    
    double x_new = gauss_angle(k_new,rng);
    

    double p = exp(k_new*(cos(x_new)+pow(x_new,2.)/2.)
                  -k_old*(cos(x_old)+pow(x_old,2.)/2.))
              *erf(pi*sqrt(k_new/2.))/erf(pi*sqrt(k_old/2.))
              *sqrt(k_old/k_new);
    
    if (UniformDouble()(rng)<p) {
        cmplx u_new = exp(1i*(x_new-arg(W_new)));
        lat.set_link(cluster.gate(),u_new);
        for (Link link : cluster.links()) {
            cmplx u = lat.s_line(link);
            lat.set_link(link,conj(u));
        }
        return 1.;
    }
    else return 0.;
}
\end{lstlisting}

\subsection*{Acceptance comparison}

\begin{figure}[!htb]
    \centering
    \import{gfx/}{cluster_acc_cont.pgf}
    \caption{Cluster algorithms acceptance contunuum limit comparison}
    \label{fig:cluster_acc_cont}
\end{figure}

\begin{figure}[!htb]
	\centering
    \import{gfx/}{cluster_side_acc.pgf}
    \caption{Cluster side acceptance comparison}
    \label{fig:cluster_side_acc}
\end{figure}

%*****************************************
%*****************************************
%*****************************************
%*****************************************
%*****************************************

%************************************************
\chapter{Toy Model}\label{ch:toy_model}
%************************************************

The definition of the $U(1)$ pure gauge theory will follow.
The lagrangian, the path integral formulation and the topological charge will be firstly defined in the continuum, and then, on the lattice.

\section{Theory in the continuum}

In a pure gauge theory, the only dynamic fields are gauge fields.
Since $U(1)$ group has only one generator, there is only one gauge field $A_\mu$.

\subsection*{Minkowski metric}

In a $1+1$ dimensional Minkowski space-time, $A_\mu$ has two components, and $\mu = 0, 1$ are respectively the time and space indices.

The classic lagrangian of the theory is:
\[
    \mathcal L[A] = -\frac{1}{4} F_{\mu\nu}F^{\mu\nu}
\]
with $F_{\mu\nu}(x) \equiv \partial_\mu A_\nu(x) - \partial_\nu A_\mu(x)$.

The $U(1)$ local gauge transformation is a symmetry of this lagrangian, in fact,
under a transformation $G(x) \equiv e^{i\theta(x)} \in U(1)$,
the gauge field transformation is:
\begin{align*}
    A_\mu(x) \rightarrow A'_\mu(x) &= G(x) A_\mu(x) G^{-1}(x) - \frac{i}{g} G(x) \partial_\mu G^{-1}(x) = \\
                                   &= A_\mu(x) - \frac{1}{g} \partial_\mu \theta(x)
\end{align*}
where $g$ is the coupling constant.

The action is: \[ S[A] = \int \mathrm d^2x\, \mathcal L[A(x)] \] and the expectation value of an operator $\widehat{\mathcal O}[A]$ is:
\[
    \left< \widehat{\mathcal O}[A] \right> = \frac{\int[\mathrm dA]\, \mathcal O[A] e^{iS[A]}}{\int[\mathrm dA]\,e^{iS[A]}}
\]
where the integrals are path integrals over field configurations.
In this formulation, field configurations are multiplied by a complex exponential.
When the intent is to compute expectation values via Monte Carlo simulations,
it is necessary to sample configurations from probability density functions.
In this theory, the integrands can be made real and positive switching the theory to euclidean metric performing a Wick rotation of the time coordinate and an analytic continuation of expectation values.

\subsection*{Euclidean metric}

Indicating with the index $M$ the objects defined in Minkowski space, the Wick rotation:
\begin{align*}
    x^M_0 &\rightarrow -i x_2 & x^M_1 &\rightarrow x_1 \\
    A^M_0 &\rightarrow i A_2 & A^M_1 &\rightarrow A_1 
\end{align*}
induce the following transformation of the action:
\[
    S^M[A] = -\frac{1}{4}\int\mathrm d^2x\,F^M_{\mu\nu}F_M^{\mu\nu} \rightarrow  S[A] = \frac{i}{4}\int\mathrm d^2x\,F_{\mu\nu}F_{\mu\nu}
\]
Thus, the expectation value of an operator $\widehat{\mathcal O}[A]$ is:
\[
    \left< \widehat{\mathcal O}[A] \right> = \frac{\int[\mathrm dA]\, \mathcal O[A] e^{-S[A]}}{\int[\mathrm dA]\,e^{-S[A]}}
\]

Here, the factor $e^{-S[A]}$ is real and positive, and can be enterpreted as a probability density function of field configurations.









%*****************************************
%*****************************************
%*****************************************
%*****************************************
%*****************************************

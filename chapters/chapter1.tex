%************************************************
\chapter{Toy Model}\label{ch:toy_model}
%************************************************

In this chapter will be defined the two dimensional $U(1)$ pure gauge theory.
Starting from the lagrangian of classic electrodynamics in a $1+1$ Minkowski space-time and its quantum field theory path integral formulation,
it will be discussed the analytic continuation to euclidean space, the definition of the topological charge and its physical interpetation.
Then, the theory will be extended to a discrete and finite lattice space, and it will be described the method used to compute physical quantities with Monte Carlo simulations.

\section{Theory in the continuum}

In a pure gauge theory, the only dynamic fields are gauge fields.
Since $U(1)$ group has only one generator, there is only one gauge field $A_\mu$.

\subsection*{Minkowski metric}

In a $1+1$ dimensional Minkowski space-time, $A_\mu$ has two components, and $\mu = 0, 1$ are respectively the time and space indices.

The classic lagrangian of the theory is:
\[
    \mathcal L[A] = -\frac{1}{4} F_{\mu\nu}F^{\mu\nu}
\]
with $F_{\mu\nu}(x) \equiv \partial_\mu A_\nu(x) - \partial_\nu A_\mu(x)$.
In natural units, the energy dimension of the gauge field is zero.

The $U(1)$ local gauge transformation is a symmetry of this lagrangian, in fact,
under a transformation $G(x) \equiv e^{i\theta(x)} \in U(1)$,
the gauge field transformation is:
\begin{align*}
    A_\mu(x) \rightarrow A'_\mu(x) &= G(x) A_\mu(x) G^{-1}(x) - \frac{i}{g} G(x) \partial_\mu G^{-1}(x) = \\
                                   &= A_\mu(x) - \frac{1}{g} \partial_\mu \theta(x)
\end{align*}
where $g$ is the coupling constant, and it has energy dimension equal to one.

The action is: \[ S[A] = \int \mathrm d^2x\, \mathcal L[A(x)] \] and the expectation value of an operator $\widehat{\mathcal O}[A]$ is:
\[
    \left< \widehat{\mathcal O}[A] \right> = \frac{\int[\mathrm dA]\, \mathcal O[A] e^{iS[A]}}{\int[\mathrm dA]\,e^{iS[A]}}
\]
where the integrals are path integrals over field configurations.
In this formulation, field configurations are multiplied by a complex exponential.
When the intent is to compute expectation values via Monte Carlo simulations,
it is necessary to sample configurations from probability density functions.
In this theory, the integrands can be made real and positive switching the theory to euclidean metric performing a Wick rotation of the time coordinate and an analytic continuation of expectation values.

\subsection*{Euclidean metric}

Indicating with the index $M$ the objects defined in Minkowski space and with $E$ the ones in the euclidean, the Wick rotation:
\begin{align*}
    &\begin{dcases}
        x^M_0 &\rightarrow -i x^E_2 \\
        x^M_1 &\rightarrow x^E_1
    \end{dcases}%
    &%
    \begin{dcases}
        A^M_0 &\rightarrow i A^E_2 \\
        A^M_1 &\rightarrow A^E_1 
    \end{dcases}&
\end{align*}
induce the following transformation of the action:
\[
    S^M[A] = -\frac{1}{4}\int\mathrm d^2x^M\,F^M_{\mu\nu}F_M^{\mu\nu} \rightarrow  S^E[A] = \frac{i}{4}\int\mathrm d^2x\,F^E_{\mu\nu}F^E_{\mu\nu}
\]
Thus, the expectation value of an operator $\widehat{\mathcal O}[A]$ is:
\begin{equation}\label{eq:cont_exp}
    \left< \widehat{\mathcal O}[A] \right> = \frac{\int[\mathrm dA]\, \mathcal O[A] e^{-S^E[A]}}{\int[\mathrm dA]\,e^{-S^E[A]}}
\end{equation}
Here, the factor $e^{-S[A]}$ is real and positive, and can be enterpreted as a probability density function of field configurations.

The Minkowski formulation will not be needed anymore in later discussions.
For this reason, every object will be considered defined in euclidean two dimensional space.

\subsection*{Topological charge}

The gauge field, expressed in a generic gauge $G(x) \equiv e^{i\theta(x)} \in U(1)$ is:
\[
    A^G_\mu(x) \equiv A_\mu(x) - \frac{1}{g}\partial_\mu\theta(x)
\]

Let $S$ be a disk in the space, $\partial S$ its boundary and $R$ its radius.
If $R \to \infty$, the gauge field, evaluated on $\partial S$,
has only the contribution induced by the gauge transformation:
\[
    \lim_{R\to\infty}A^G_\mu(x) = -\frac{1}{g}\partial_\mu\theta(x)
\]
and its circulation integral over the border is:
\[
    \lim_{R\to\infty}\oint\limits_{\partial S}\mathrm dl_\mu\,A^G_\mu(x) = -\frac{1}{g}\oint\limits_{\partial S}\mathrm dl_\mu\partial_\mu\theta(x)
\]

The definition of $\theta(x)$ leaves a further unfixed constant term:
\[
    \theta(x) \rightarrow \theta(x) + 2\pi n, \quad \text{with}\ n \in \mathbb Z
\]
thus, being $\theta(x)$ a conservative function, only the constant term contributes to the circulation integral:
\[
    \lim_{R\to\infty}\oint\limits_{\partial S}\mathrm dl_\mu\,A^G_\mu(x) = \frac{2\pi}{g} Q, \quad \text{with}\ Q \in \mathbb Z \\
\]
where $Q$ is the topological charge.

Applying then Stokes theorem:
\begin{align*}
    Q &= \frac{g}{2\pi}\lim_{R\to\infty}\oint\limits_{\partial S}\mathrm dl_\mu\,A^G_\mu(x) = \\
      &= \frac{g}{2\pi}\lim_{R\to\infty}\int\limits_S\mathrm d\vv s\,\cdot \vv\nabla \times \vv A^G(x) = \\
      &= \frac{g}{2\pi}\lim_{R\to\infty}\int\limits_S\mathrm d^2x\, \left(\partial_1 A^G_2(x) - \partial_2 A^G_1(x)\right) = \\
      &= \frac{g}{2\pi}\lim_{R\to\infty}\int\limits_S\mathrm d^2x\, F^G_{12}(x) = \\
      &= \frac{g}{4\pi}\lim_{R\to\infty}\int\limits_S\mathrm d^2x\, \epsilon_{\mu\nu}F^G_{\mu\nu}(x)
\end{align*}
the definition of the topological charge density becomes clear:
\[
    q(x) \equiv \frac{g}{4\pi}\epsilon_{\mu\nu}F_{\mu\nu}(x)
\]
and its integral over all the space is the topological charge is:
\[
    Q = \int\mathrm d^2x\,q(x)
\]

Finally, being $\vv B(x) = \vv\nabla \times \vv A(x)$, the charge density is proportional to the flux of the magnetic field across an infinitesimal surface,
and the total charge is proportional to the total flux.

\section{Lattice formulation}
The goal is to compute physical observables numerically, and the expectation value \eqref{eq:cont_exp} provide a way to do that.
However, both the numerator and the denominator are divergent in the continuum, and it is necessary an integral regularization to isolate the divergent term,
and hence remove it.

The idea of lattice regularization is to define a theory on a discrete space time lattice with the requirement that,
when the lattice becomes finer, the lattice theory approaches to the continuum theory.

The path integrals are then approximated with finite dimensional integrals,
and the continuum expectation value \eqref{eq:cont_exp} is obtained as the continuum limit of the lattice approximated expectation value.

It is then necessary to provide new definition for the action and the topological charge on the lattice,
and they need to converge to the continuum definition when the lattice spacing approaches zero.

Once the lattice theory is defined, it remains to describe how to compute expectation values with Monte Carlo simulations, and how to extrapolate the continuum limit.

%\subsection*{Dimensionless quantities}
%Numerical studies are usually performed with dimensionless parameters because it simplifies result comparisons and the implementation.
%For this reason, it is clearer if this convention is adopted from the beginning of the lattice formulation.
%
%Let $a$ be the lattice spacing in both direction. It has energy dimension $[a] = -1$ and it can be used to define dimensionless quantities.

\subsection*{Gauge invariant objects}

It is convenient to define an action that is already exactly gauge invariant, even when far from the continuum limit.
By doing so, not only it prevents a further bias to be added when extrapolating the continuum limit,
but it is also possible to apply gauge transformations of the fields without modifying the energy of the system.
It will be shown that the latter property is crucial for the performance of the cluster algorithm described in Chapter \ref{ch:cluster}.

To easily implement gauge invariance, it is useful to encapsulate gauge fields inside \emph{Wilson lines}, which are defined over continuous paths in the space.
Let $\mathcal C(x,y)$ be a path that goes from $x$ to $y$. The correspondent Wilson line is:
\[
    U[\mathcal C(x,y)] = e^{ig\int_{\mathcal C(x,y)}\mathrm dz_\mu\,A_\mu(z)} \in U(1)
\]
Under a gauge transformation $G(x) \equiv e^{i\theta(x)}$, the \emph{Wilson line} transformation is:
\begin{equation}\label{eq:path_gauge}
    \begin{aligned}
    U[\mathcal C(x,y)] \rightarrow U'[\mathcal C(x,y)] &= e^{ig\int_{\mathcal C(x,y)}\mathrm dz_\mu\,\left(A_\mu(z) - \frac{1}{g}\partial_\mu\theta(x)\right)} = \\
                                                       &= U[\mathcal C(x,y)]\,e^{i(\theta(x) - \theta(y))} = \\
                                                       &= G(x)U[\mathcal C(x,y)]G^{-1}(y)
    \end{aligned}
\end{equation}
If the path is closed, the the correspondent Wilson line, \ie a \emph{Wilson loop}, is gauge invariant:
\[
    U[\mathcal C(x,x)] \rightarrow U[\mathcal C(x,x)]
\]

Wilson loops are then the required invariant objects since they can be constructed also on a discrete space,
and the lattice gauge invariant action will be expressed in terms of them.

\subsection*{Lattice sites, links and plaquettes}
Let the two dimensional space be discretized into a squared grid of points, \ie a lattice, and let $a$ be the side of the squares.
A grid point is usually called \emph{site} and its coordinates labelled with an index $n$.
The site of coordinates $n + \hat\mu$ with $\mu \in {1,2}$ corresponds to the point that has the $\mu$th coordinate increased by $a$.

The most elementary path that can be defined on the lattice is a \emph{link},
\ie a straight line of length $a$ that connects two adjacent sites (Figure \ref{fig:links}).
\begin{figure}[!htb]
    \centering
    \rule{3in}{1.5in}
    \caption{Wilson lines over links}
    \label{fig:links}
\end{figure}
The Wilson line evaluated over the link that connects $n$ to $n+\hat\mu$ is:
\begin{equation}\label{eq:link}
    \begin{aligned}
        U_{\hat\mu}(n) &\equiv e^{ig\int_n^{n+\hat\mu}\mathrm dz_\mu\,A_\mu(z)} = \\
                       &= e^{ig\int_n^{n+\hat\mu}\mathrm dz_\mu\,\left[A_\mu(n+\hat\mu/2) + (z_\mu-n-\hat\mu/2)\partial_\mu A_\mu(n+\hat\mu/2)%
                                                                                     + \mathcal O\left((z_\mu-n-\hat\mu/2)^2\right)\right]} = \\
                       &= e^{igaA_\mu(n+\hat\mu/2) + \mathcal O\left(a^3\right)}
    \end{aligned}
\end{equation}
Thus, in the limit $a \to 0$, it is related to the gauge field evaluated at the midpoint of the link.
These link Wilson lines are also called \emph{link variables}, and the lattice action will be expressed in terms of them,
encapsulating the gauge fields inside link variables.

If the link path is evaluated in the opposite direction, the value is its complex conjugate (or its inverse):
\[
    U_{-\hat\mu}(n+\hat\mu) = e^{ig\int^n_{n+\hat\mu}\mathrm dz_\mu\,A_\mu(z)} = e^{-ig\int_n^{n+\hat\mu}\mathrm dz_\mu\,A_\mu(z)} = U^*_{\hat\mu}(n)
\]
For a finite lattice, the number of \emph{independent} link variables is then equal to the number of sites multiplied by the number of dimensions.

A local gauge transform $G(n) \in U(1)$ applied to site $n$ affects all links connected to it.
Using Equation \eqref{eq:path_gauge}, the transformation of link variables is:
\begin{equation}\label{eq:site_gauge}
    \begin{aligned}
        &\begin{dcases}
        U_\mu(n) \rightarrow G(n)U_\mu(n) \\
        U_{-\mu}(n) \rightarrow G(n)U_{-\mu}(n)
        \end{dcases}%
        &%
        \begin{dcases}
        U_\mu(n-\hat\mu) \rightarrow U_\mu(n-\hat\mu)G^*(n) \\
        U_{-\mu}(n+\hat\mu) \rightarrow U_{-\mu}(n+\hat\mu)G^*(n)
        \end{dcases}&
    \end{aligned}
\end{equation}
This means that if a link variable is coming from site $n$, it will be multiplied by $G(n)$, but,
if it is going toward site $n$, it will be multiplied by $G^*(n)$.
This kind of gauge transformations is very important for this work because the cluster algorithm described in Chapter \ref{ch:cluster}
will perform a sequence of them before cluster inversion.

Any Wilson line on the lattice can be expressed in terms of link variables,
in fact, decomposing a lattice line $\ell$ as a sequence $\ell_1, \ell_2, \ldots$ of links (Figure \ref{fig:line}),
the Wilson line evaluated over $\ell$ is simply given by the product of the correspondent links variables:
\begin{figure}[!htb]
    \centering
    \rule{3in}{1.5in}
    \caption{Generic lattice Wilson line}
    \label{fig:line}
\end{figure}
\[
    prova
\]

It remains to define the elementary gauge invariant object: the \emph{plaquette} Wilson loop.
A lattice plaquette is a grid square. Its sides are four links and its corners are four sites.
The shortest possible Wilson loop is a closed loop around a plaquette.
It has a very important physical relevance, and, to understand why,
it is necessary to express it in terms of the gauge fields.
Considering a link loop that starts from site $s$ and runs anticlockwise (Figure \ref{fig:plaq}),
the corresponding Wilson line is:
\begin{figure}[!htb]
    \centering
    \rule{3in}{1.5in}
    \caption{Plaquette Wilson loop}
    \label{fig:plaq}
\end{figure}
\begin{equation}\label{eq:plaq}
    \begin{aligned}
        \Pi_{\hat x\hat y}(s) &\equiv U_{\hat x}(s) U_{\hat y}(s+\hat x) U_{-\hat x}(s+\hat x+\hat y) U_{-\hat y}(s+\hat y) = \\
                              &= U_{\hat x}(s) U_{\hat y}(s+\hat x) U^*_{\hat x}(s+\hat y) U^*_{\hat y}(s) = \\
                              &= e^{iga[A_1(s+\hat x/2) + A_2(s+\hat x+\hat y/2) - A_1(s+\hat x/2+\hat y) - A_2(s+\hat y/2)]+ \mathcal O\left(a^3\right)} = \\
                              &= e^{iga^2[\partial_1A_1(s)/2 + \partial_1A_2(s) + \partial_2A_2(s)/2 - \partial_1A_1(s)/2 - \partial_2A_1(s) - \partial_2A_2(s)/2] %
                                    + \mathcal O\left(a^3\right)} = \\
                              &= e^{iga^2F_{12}(s) + \mathcal O\left(a^3\right)}
    \end{aligned}
\end{equation}
If $a \to 0$, from the plaquette Wilson loop starting at site $s$, is then possible to recover the gauge field tensor evaluated at $s$.

\subsection*{Lattice action}

Equation \eqref{eq:plaq} 

\subsection*{Lattice topological charge}

\subsection*{Monte Carlo simulation}

\subsection*{Continuum limit}







%*****************************************
%*****************************************
%*****************************************
%*****************************************
%*****************************************

%************************************************
\chapter{Toy Model}\label{ch:toy_model}
%************************************************

In this chapter the two dimensional $U(1)$ pure gauge theory will be introduced.
Starting from the Lagrangian of classic electrodynamics in a $1+1$ Minkowski space-time and its path integral formulation,
it will be discussed the analytic continuation to euclidean space, the definition of the topological charge and its physical interpretation.
Then, the theory will be extended to a discrete lattice space, and it will be described the method used to compute physical quantities with Monte Carlo simulations.

\tikzset{->-/.style={decoration={markings,mark=at position .55 with {\arrow{to}}},postaction={decorate}}} 

\section{Theory in the continuum}

In a pure gauge theory, the only dynamic fields are gauge fields.
Since the $U(1)$ group has only one generator, there is only one gauge field $A_\mu$.

\subsection*{Minkowski metric}

In a $1+1$ dimensional Minkowski space-time, $A_\mu$ has two components, where $\mu = 0, 1$ are respectively the time and space indices.

The classic lagrangian of the theory is:
\[
    \mathcal L[A] = -\frac{1}{4} F_{\mu\nu}F^{\mu\nu}
\]
Where $F_{\mu\nu} \equiv \partial_\mu A_\nu - \partial_\nu A_\mu$ is the gauge field tensor.
In natural units, the energy dimension of the gauge field is zero.

The $U(1)$ local gauge transformation is a symmetry of this Lagrangian, indeed,
under a transformation $G(x) \equiv e^{i\theta(x)} \in U(1)$,
the gauge field transformation is:
\begin{align*}
    A_\mu(x) \rightarrow A'_\mu(x) &= G(x) A_\mu(x) G^{-1}(x) - \frac{i}{g} G(x) \partial_\mu G^{-1}(x) \\
                                   &= A_\mu(x) - \frac{1}{g} \partial_\mu \theta(x)
\end{align*}
and the gauge field tensor is invariant since $\partial_\nu\partial_\mu\theta(x) = \partial_\mu\partial_\nu\theta(x)$:
\[
    F_{\mu\nu}(x) \rightarrow F'_{\mu\nu}(x) = F_{\mu\nu}(x)
\]
$g$ is the coupling constant, and it has energy dimension equal to one.

The action is:
\[ S[A] = \int \mathrm d^2x\, \mathcal L[A(x)]
\]
and the time ordered vacuum expectation value of an operator $\widehat{\mathcal O}[A]$ is:
\begin{equation}\label{eq:min_exp}
    \left< \widehat{\mathcal O}[A] \right> = \frac{\int[\mathrm dA]\, \mathcal O[A] e^{iS[A]}}{\int[\mathrm dA]\,e^{iS[A]}}
\end{equation}
where the integrals are path integrals over field configurations.
The integral ratio can be interpreted as a weighted mean of $\mathcal O[A]$ evaluated at all possible configurations, weighted by the function $e^{iS[A]}$.
However, in this case, the weight function is oscillatory, and it is not possible to define the measure over field configurations.

Within this theory, the weight function can be made a negative real exponential switching the theory to Euclidean metric performing a Wick rotation of the time coordinate.
In this way, the measure can be defined as a Wiener measure and the expectation value of Equation \eqref{eq:min_exp} can then be obtained by analytic continuation.

The Euclidean path integral formulation is also a valid stand-alone physical theory, indeed,
it provides expectation values of operators at thermodynamic equilibrium, and hence vacuum expectation values.

Furthermore, a real and positive weight function can be interpreted as a probability density function of field configurations,
and this fact is of crucial importance if a path integral has to be evaluated with the Monte Carlo method.

\subsection*{Euclidean metric}

Indicating with the index $M$ the objects defined in Minkowski space and with $E$ the ones in the Euclidean, the Wick rotation:
\begin{align*}
    &\begin{dcases}
        x_0^M &\rightarrow -i x_2^E \\
        x_1^M &\rightarrow x_1^E
    \end{dcases}%
    &%
    \begin{dcases}
        A^M_0 &\rightarrow i A^E_2 \\
        A^M_1 &\rightarrow A^E_1 
    \end{dcases}&
\end{align*}
induces the following transformation of the action:
\[
    S^M[A] = -\frac{1}{4}\int\mathrm d^2x^M\,F^M_{\mu\nu}F_M^{\mu\nu} \rightarrow  \frac{i}{4}\int\mathrm d^2x\,F^E_{\mu\nu}F^E_{\mu\nu}
\]
Thus, with the following definitions:
\begin{equation}\label{eq:cont_action}
    \begin{aligned}
        \mathcal L^E &\equiv \frac{1}{4}F^E_{\mu\nu}F^E_{\mu\nu} \\
        S^E[A] &\equiv \int\mathrm d^2x\,\mathcal L^E[A(x)]
    \end{aligned}
\end{equation}
the expectation value of an operator $\widehat{\mathcal O}[A]$ is:
\begin{equation}\label{eq:cont_exp}
    \left< \widehat{\mathcal O}[A] \right> = \frac{\int[\mathrm dA]\, \mathcal O[A] e^{-S^E[A]}}{\int[\mathrm dA]\,e^{-S^E[A]}}
\end{equation}
Here, the factor $e^{-S[A]}$ is real and positive, and can be enterpreted as a probability density function over field configurations.

The Minkowski formulation will not be needed anymore in later discussions.
For this reason, every object will be considered defined in a euclidean two dimensional space.

\subsection*{Topological charge}

The gauge field, expressed in a generic gauge $G(x) \equiv e^{i\theta(x)} \in U(1)$ is:
\[
    A^G_\mu(x) \equiv A_\mu(x) - \frac{1}{g}\partial_\mu\theta(x)
\]

Let $S$ be a disk in the space, $\partial S$ its boundary and $R$ its radius.
In an infinite space, the gauge field $A_\mu$ is asymptotically approaching zero.
Thus, If $R \to \infty$, the gauge field expressed in a gauge $G$, evaluated on $\partial S$,
has only the contribution induced by the gauge transformation:
\[
    \lim_{R\to\infty}A^G_\mu(x) = -\frac{1}{g}\partial_\mu\theta(x)
\]
and its circulation integral over the border is:
\begin{equation}\label{eq:circulation}
    \lim_{R\to\infty}\oint\limits_{\partial S}\mathrm dl_\mu\,A^G_\mu(x) = -\frac{1}{g}\oint\limits_{\partial S}\mathrm dl_\mu\partial_\mu\theta(x)
\end{equation}
If $\theta(x)$ is differentiable, then Equation \eqref{eq:circulation} is equal to zero.
However, $\theta(x)$, which is the complex argument of $G(x)$, is discontinuous along $\partial S$.
Indeed, in order to have a single valued complex argument function, it is necessary to confine it in a $2\pi$ sized interval, \ie $(-\pi,\pi]$.
For this reason, the integral of $\partial_\mu\theta(x)$ along $\partial S$ receives a contribution of $\pm 2\pi$ every time $\theta(x)$ runs across a discontinuity,
%However, the definition of $\theta(x)$ leaves a further unfixed constant term:
%\[
%    \theta(x) \rightarrow \theta(x) + 2\pi n, \quad \text{with}\ n \in \mathbb Z
%\]
%and this redundancy can be fixed imposing:
%\[
%    \theta(x) \in (-\pi,\pi]
%\]
%Which makes $\theta(x)$ not continuous.
%
%Along the path $\partial S$, every time $\theta(x)$ increases and passes across the superior border at $\pi$, it acquires a new value,
%just above the inferior border $-\pi$.
%On the contrary, when $\theta(x)$ decreases so that it would reach $-\pi$, it becomes $\pi$ instead.
%
%When integrating $\partial_\mu\theta(x)$ over these discontinuities,
%they contribute with $-2\pi$ if $\theta(x)$ pass across the superior border, and with $2\pi$ if across the inferior.
%Apart from these discontinuity points, the function $\partial_\mu\theta(x)$ is conservative, thus,
%there is not any other contribution to the circulation apart from the ones that are due to the discontinuities.
%
%Letting $Q_+$ and $Q_-$ be respectively the number of times $\theta(x)$ cuts the superior or the inferior extreme of the interval,
and the result of the circulation of Equation \eqref{eq:circulation} is:
\[
    \lim_{R\to\infty}\oint\limits_{\partial S}\mathrm dl_\mu\,A^G_\mu(x) = \frac{2\pi}{g} Q
\]
where $Q \in \mathbb Z$ is the topological charge.
%, \ie the net number of times $\theta(x)$ has wrapped around the interval $(-\pi,\pi]$,
%and $Q$ is positive if the winding has happened from the positive direction, while it is negative if from the other side.

Applying Stokes theorem:
\begin{align*}
    Q &= \frac{g}{2\pi}\lim_{R\to\infty}\oint\limits_{\partial S}\mathrm dt_\mu\,A^G_\mu(t) \\
      &= \frac{g}{2\pi}\lim_{R\to\infty}\int\limits_S\mathrm d\vv s\,\cdot \vv\nabla \times \vv A^G(\vv s) \\
      &= \frac{g}{2\pi}\lim_{R\to\infty}\int\limits_S\mathrm d^2x\, \left(\partial_1 A^G_2(x) - \partial_2 A^G_1(x)\right) \\
      &= \frac{g}{2\pi}\lim_{R\to\infty}\int\limits_S\mathrm d^2x\, F^G_{12}(x) \\
      &= \frac{g}{4\pi}\lim_{R\to\infty}\int\limits_S\mathrm d^2x\, \epsilon_{\mu\nu}F^G_{\mu\nu}(x)
\end{align*}
the definition of the topological charge density becomes then clear:
\begin{equation}\label{eq:top_charge_density}
    q(x) \equiv \frac{g}{4\pi}\epsilon_{\mu\nu}F_{\mu\nu}(x)
\end{equation}
and its integral over all the space is the topological charge:
\begin{equation}\label{eq:top_charge}
    Q \equiv \int\mathrm d^2x\,q(x)
\end{equation}

Finally, being $\vv B(x) = \vv\nabla \times \vv A(x)$, the charge density is proportional to the flux of the magnetic field across an infinitesimal surface,
and the total charge is proportional to the total flux.

The topological charge is integer valued also if the space considered is a compact and orientable manifold.
%It is possible to obtain an integer valued topological charge also in a finite space,
%is a compact and orientable manifold.
Indeed, the circulation of the vector potential $A_\mu$ over $\partial S$ is equal to the negative of the circulation over $-\partial S$,
which is also the flux of the magnetic field across the complementary of $S$.
Since these two fluxes have to be the same, the circulation of $A_\mu$ has to be:
\[
    \oint\limits_{\partial S}\mathrm dl_\mu\,A_\mu(x) = 0
\]
leaving only the gauge function part as before:
\[
    \oint\limits_{\partial S}\mathrm dl_\mu\,A_\mu^G(x) = -\frac{1}{g}\oint\limits_{\partial S}\mathrm dl_\mu\,\partial_\mu\theta(x)
\]
Thus, the topological charge is always integer-valued also on a compact and orientable manifold.

\section{Lattice formulation}
The goal is to compute physical observables numerically, and the form of Equation \eqref{eq:cont_exp} suggests a way to do that.
However, both the numerator and the denominator are divergent in the continuum, and it is necessary an integral regularization to isolate the divergent term,
and hence remove it.

The idea of lattice regularization is to define a theory on a discrete space-time grid of points, \ie a lattice,
with the requirement that, when the grid spacing becomes finer and finer,
the lattice theory approaches the continuum theory.

The path integrals are then approximated with multidimensional integrals, in which the integration variables are equal to the number of lattice points.
The continuum expectation value \eqref{eq:cont_exp} is then obtained as the continuum limit of the lattice approximated expectation value.

It is then necessary to provide new definitions for the action and the topological charge on the lattice,
and they need to converge to the continuum definition if the lattice spacing approaches zero.

Once the lattice theory is defined, it remains to describe how to compute expectation values with Monte Carlo simulations, and how to extrapolate to the continuum limit.

\subsection*{Gauge invariant objects}

It is convenient to define an action that is already exactly gauge invariant, even when the space is discrete and far from being continuous.
By doing so, not only it prevents a further bias to be added when extrapolating the continuum limit,
but it is also possible to apply gauge transformations of the fields without modifying the energy of the system.
It will be shown that this latter property is crucial for the performance of the cluster algorithm described in Chapter \ref{ch:cluster}.

To easily implement gauge invariance, it is useful to encapsulate gauge fields inside \emph{Schwinger lines}, which are defined over continuous paths in the space.
Let $\mathcal C(x,y)$ be a path that goes from $x$ to $y$. The correspondent Schwinger line is:
\[
    U[\mathcal C(x,y)] \equiv e^{ig\int_{\mathcal C(x,y)}\mathrm dt_\mu\,A_\mu(t)} \in U(1)
\]
Under a gauge transformation $G(x) \equiv e^{i\theta(x)}$, the Schwinger line transformation is:
\begin{equation}\label{eq:path_gauge}
    \begin{aligned}
    U[\mathcal C(x,y)] \rightarrow U'[\mathcal C(x,y)] &= e^{ig\int_{\mathcal C(x,y)}\mathrm dt_\mu\,\left(A_\mu(t) - \frac{1}{g}\partial_\mu\theta(t)\right)} \\
                                                       &= U[\mathcal C(x,y)]\,e^{i(\theta(x) - \theta(y))} \\
                                                       &= G(x)U[\mathcal C(x,y)]G^{-1}(y)
    \end{aligned}
\end{equation}
Thus, if the path is closed, the corresponding Schwinger line, \ie a \emph{Wilson loop}, is gauge invariant:
\[
    U[\mathcal C(x,x)] \rightarrow G(x)U[\mathcal C(x,x)]G^{-1}(x) = U[\mathcal C(x,x)]
\]

Wilson loops are then the required invariant objects, since they can be constructed also on a discrete space,
and the lattice gauge invariant action will be expressed in terms of them.

\subsection*{Lattice sites, links and plaquettes}
Let the two dimensional space be discretized into a squared lattice, and let $a$ be the side of the squares.
A grid point is usually called \emph{site} and its coordinates will be labelled with an index $s$.
The unit of length of lattice coordinates is the lattice spacing $a$.
The site $s + \hat\mu$, with $\hat\mu = \hat1,\hat2$, corresponds to the point that has the $\mu^\mathrm{th}$ coordinate increased by one.

The most elementary path that can be defined on the lattice is a \emph{link},
\ie a straight line of length $a$ that connects two adjacent sites.
The link that connects $s$ to $s+\hat\mu$ will be called $\mathcal L_\mu(s)$ (Figure \ref{fig:links}).
\begin{figure}[!htb]
    \centering
    \begin{tikzpicture}[x=1cm,y=1cm]
        \draw[step=1, help lines, dashed, color=black!30] (0,0) grid (5.5,3.5);
        \draw[->,thick,black] (0,0) -- (5.5,0) node [right] {$x_1$};
        \draw[->,thick,black] (0,0) -- (0,3.5) node [above] {$x_2$};

        \draw[->-,thick,RoyalBlue] (1,1) -- node [black,yshift=0.4cm] {$\mathcal L_1(1,1)$} (2,1);
        \draw[->-,thick,RoyalBlue] (3,2) node [black,above] {$\mathcal L_{-2}(3,2)$} --  (3,1);
    \end{tikzpicture}
    \caption{Lattice links}
    \label{fig:links}
\end{figure}

The Wilson line evaluated over $\mathcal L_\mu(s)$ is:
\begin{equation}\label{eq:link}
    \begin{aligned}
        U[\mathcal L_{\mu}(s)] &\equiv e^{ig\int_s^{s+\hat\mu}\mathrm dt_\mu\,A_\mu(t)} \\
                       &= e^{ig\int_s^{s+\hat\mu}\mathrm dt_\mu\,\left[A_\mu(s) + \mathcal O\left(t_\mu-s\right)\right]} \\
                       &= e^{iga\left[A_\mu(s) + \mathcal O\left(a\right)\right]}
    \end{aligned}
\end{equation}
Thus, in the limit $a \to 0$, it is related to the gauge field evaluated at site $s$.
Wilson lines over links are also called \emph{link variables}, and the lattice action will be expressed in terms of them,
encapsulating the gauge fields inside link variables.

If the link path is evaluated in the opposite direction, the value is its complex conjugate (or its inverse):
\[
    U[\mathcal L_{-\mu}(n+\hat\mu)] = e^{ig\int^n_{n+\hat\mu}\mathrm dt_\mu\,A_\mu(t)}
                                        = e^{-ig\int_n^{n+\hat\mu}\mathrm dt_\mu\,A_\mu(t)} = U^*[\mathcal L_{\mu}(n)]
\]
For a finite lattice, the number of \emph{independent} link variables is then equal to the number of links that are present,
\ie the number of sites multiplied by the number of dimensions.

If a local gauge transformation $G(s) \in U(1)$ is applied to site $s$, it affects all link variables connected to it.
Using Equation \eqref{eq:path_gauge}, the transformation of link variables are (Figure \ref{fig:local_gauge}):
\begin{equation}\label{eq:site_gauge}
    \begin{aligned}
        &\begin{dcases}
            U[\mathcal L_\mu(s)] \rightarrow G(s)U[\mathcal L_\mu(s)] \\
            U[\mathcal L_{-\mu}(s)] \rightarrow G(s)U[\mathcal L_{-\mu}(s)]
        \end{dcases} \\
        &\begin{dcases}
            U[\mathcal L_\mu(s-\hat\mu)] \rightarrow U[\mathcal L_\mu(s-\hat\mu)]G^*(s) \\
            U[\mathcal L_{-\mu}(s+\hat\mu)] \rightarrow U[\mathcal L_{-\mu}(s+\hat\mu)G^*(s)]
        \end{dcases}
    \end{aligned}
\end{equation}

\begin{figure}[!htb]
    \centering
    \begin{tikzpicture}[x=1.5cm,y=1.5cm]
        \draw[step=1, help lines, dashed, color=black!30] (0,0) grid (2.5,2.5);
        \draw[->,thick,black] (0,0) -- (2.5,0) node [right] {$x_1$};
        \draw[->,thick,black] (0,0) -- (0,2.5) node [above] {$x_2$};

        \draw[->-,thick,RoyalBlue] (1,1) -- node [black,yshift=0.4cm] {$U_1$} (2,1);
        \draw[->-,thick,RoyalBlue] (1,1) -- node [black,left] {$U_2$} (1,2);
        \draw[->-,thick,RoyalBlue] (1,1) -- node [black,yshift=-0.4cm] {$U_3$} (0,1);
        \draw[->-,thick,RoyalBlue] (1,1) -- node [black,right] {$U_4$} (1,0);

        \draw[->,thick,black] (3,1) -- node [black,yshift=0.4cm] {$G$} (4,1);

        \draw[step=1, help lines, dashed, color=black!30, xshift=0.75cm] (4,0) grid (6.5,2.5);
        \draw[->,thick,black] (4.5,0) -- (7,0) node [right] {$x_1$};
        \draw[->,thick,black] (4.5,0) -- (4.5,2.5) node [above] {$x_2$};

        \draw[->-,thick,RoyalBlue] (5.5,1) -- node [black,yshift=0.4cm] {$GU_1$} (6.5,1);
        \draw[->-,thick,RoyalBlue] (5.5,1) -- node [black,left] {$GU_2$} (5.5,2);
        \draw[->-,thick,RoyalBlue] (5.5,1) -- node [black,yshift=-0.4cm] {$GU_3$} (4.5,1);
        \draw[->-,thick,RoyalBlue] (5.5,1) -- node [black,right] {$GU_4$} (5.5,0);
    \end{tikzpicture}
    \caption{Local gauge transform of link variables}
    \label{fig:local_gauge}
\end{figure}
This means that if a link variable is coming from site $s$, it will be multiplied by $G(s)$, but,
if it is going toward site $s$, it will be multiplied by $G^*(s)$.
This kind of gauge transformations are very important for this work, because the cluster algorithm described in Chapter \ref{ch:cluster}
will perform a sequence of them before being ready for cluster inversion.


A generic lattice path consists in a polygonal chain of links (Figure \ref{fig:path}),
\begin{figure}[!htb]
    \centering
    \begin{tikzpicture}[x=1cm,y=1cm]
        \draw[step=1, help lines, dashed, color=black!30] (0,0) grid (6.5,3.5);
        \draw[->,thick,black] (0,0) -- (6.5,0) node [right] {$x_1$};
        \draw[->,thick,black] (0,0) -- (0,3.5) node [above] {$x_2$};

        \draw[->-,thick,RoyalBlue] (1,2) -- node [black,left] {$\mathcal L_1$} (1,1);
        \draw[->-,thick,RoyalBlue] (1,1) -- node [black,yshift=0.4cm] {$\mathcal L_2$} (2,1);
        \draw[->-,thick,RoyalBlue] (2,1) -- node [black,yshift=0.4cm] {$\mathcal L_3$} (3,1);
        \draw[->-,thick,RoyalBlue] (3,1) -- node [black,right] {$\mathcal L_4$} (3,2);
        \draw[->-,thick,RoyalBlue] (3,2) -- node [black,yshift=0.4cm] {$\mathcal L_5$} (4,2);
        \draw[->-,thick,RoyalBlue] (4,2) -- node [black,yshift=0.4cm] {$\mathcal L_6$} (5,2);
        \draw[->-,thick,RoyalBlue] (5,2) -- node [black,right] {$\mathcal L_7$} (5,1);
    \end{tikzpicture}
    \caption{A generic lattice path $\mathcal C$}
    \label{fig:path}
\end{figure}
and the corresponding Wilson line can be expressed in terms of the link variables.
In fact, calling $\mathcal C$ the generic path, and $\mathcal L_1, \mathcal L_2, \ldots$ the links that form it,
the Wilson line evaluated over $\mathcal C$ is the product of the link variables:
\begin{align*}
    U[\mathcal C] &= e^{ig\int_{\mathcal C}\mathrm dt_\mu\,A_\mu(t)} \\
            &= e^{ig\left[\int_{\mathcal L_1}\mathrm dt_\mu\,A_\mu(t) + \int_{\mathcal L_2}\mathrm dt_\mu\,A_\mu(t) + \cdots \right]} \\
            &= U[\mathcal L_1]U[\mathcal L_2] \cdots
\end{align*}

It remains to define the elementary gauge invariant object: the \emph{plaquette} Wilson loop.
A lattice plaquette is a grid square, and the shortest possible Wilson loop is a closed loop around a plaquette.
It has a very important physical relevance, and, to understand why,
it is necessary to express it in terms of the gauge fields.

\begin{figure}[!htb]
    \centering
    \begin{tikzpicture}[x=1.5cm,y=1.5cm]
        \draw[step=1, help lines, dashed, color=black!30] (0,0) grid (5.5,2.5);
        \draw[->,thick,black] (0,0) -- (5.5,0) node [right] {$x_1$};
        \draw[->,thick,black] (0,0) -- (0,2.5) node [above] {$x_2$};

        \draw[->-,thick,RoyalBlue] (2,1) -- node [black,yshift=-0.5cm] {$\mathcal L_1(s)$} (3,1);
        \draw[->-,thick,RoyalBlue] (3,1) -- node [black,right] {$\mathcal L_{2}(s+\hat1)$} (3,2);
        \draw[->-,thick,RoyalBlue] (3,2) -- node [black,yshift=0.5cm] {$\mathcal L_{-1}(s+\hat1+\hat2)$} (2,2);
        \draw[->-,thick,RoyalBlue] (2,2) -- node [black,left] {$\mathcal L_{-2}(s+\hat2)$} (2,1);
    \end{tikzpicture}
    \caption{Plaquette loop $\mathcal P$}
    \label{fig:plaq}
\end{figure}
Let $\mathcal P_{12}(s)$ be the path that starts from site $s$ and runs counter-clockwise around the plaquette (Figure \ref{fig:plaq}).
Computing the corresponding Wilson loop using Equation \eqref{eq:link}:
\begin{equation*}
    \begin{aligned}
        U[\mathcal P_{12}(s)] &\equiv U[\mathcal L_{1}(s)] U[\mathcal L_{2}(s+\hat 1)] U[\mathcal L_{-1}(s+\hat 1+\hat 2)] U[\mathcal L_{-2}(s+\hat 2)] \\
                              &= U[\mathcal L_{1}(s)] U[\mathcal L_{2}(s+\hat 1)] U^*[\mathcal L_{1}(s+\hat 2)] U^*[\mathcal L_{2}(s)] \\
                              &= e^{iga[A_1(s) + A_2(s+\hat 1) - A_1(s+\hat 2) - A_2(s) + \mathcal O\left(a\right)]} \\
                              &= e^{iga[a\partial_1A_2(s) - a\partial_2A_1(s) + \mathcal O\left(a\right)]} \\
                              &= e^{iga\left[aF_{12} + \mathcal O\left(a\right)\right]}
    \end{aligned}
\end{equation*}
Here, the former leading order has vanished, and the error is of the same order as the new leading order.
To get a clean result it is necessary to expanding further Equation \eqref{eq:link}:
\begin{equation*}%\label{eq:link}
    \begin{aligned}
        U[\mathcal L_{\mu}(s)] &\equiv e^{ig\int_s^{s+\hat\mu}\mathrm dt_\mu\,A_\mu(t)} \\
                       &= e^{ig\int_s^{s+\hat\mu}\mathrm dt_\mu\,\left[A_\mu(s) + (t_\mu-s)\partial_\mu A_\mu(s) + \mathcal O\left((t_\mu-s)^2\right)\right]} \\
                       &= e^{iga\left[A_\mu(s) + \frac{a}{2}\partial_\mu A_\mu(s) + \mathcal O\left(a^2\right)\right]} \\
    \end{aligned}
\end{equation*}
Thus, the leading order is:
\begin{equation}\label{eq:plaq}
    \begin{aligned}
        U[\mathcal P_{12}(s)] &= e^{iga^2[F_{12} + \frac{1}{2}(\partial_1A_1(s) + \partial_2A_2(s) - \partial_1A_1(s) - \partial_2A_2(s)) %
                                   + \mathcal O\left(a\right)]} \\
                         &= e^{iga^2\left[F_{12}(s) + \mathcal O\left(a\right)\right]}
    \end{aligned}
\end{equation}
If $a \to 0$, it is then related the gauge field tensor evaluated at site $s$.

\subsection*{Lattice action}

The continuum action \eqref{eq:cont_action}, depends quadratically on the gauge field tensor.
In order recover this dependence, it is useful to take the real part of the plaquette loop of Equation \eqref{eq:plaq} and expand it:
\begin{align*}
    \Re\,U[\mathcal P_{12}(s)] &= \cos\left[ga^2F_{12}(s) + \mathcal O\left(a^3\right)\right] \\
                          &= 1 - \frac{1}{2}g^2a^4F_{12}(s)^2 + \mathcal O\left(a^5\right) \\
                          &= 1 - \frac{1}{4}g^2a^4F_{\mu\nu}(s)F_{\mu\nu}(s) + \mathcal O\left(a^5\right)
\end{align*}
Indeed, defining the lattice action in the following form:
\begin{equation}\label{eq:lat_action}
    \begin{aligned}
        S^L[U] \equiv \beta\sum_s\left(1-\Re\,U[\mathcal P_{12}(s)]\right)
    \end{aligned}
\end{equation}
and using $a^2\sum_s \xrightarrow{a\to0} \int\mathrm d^2x$, the infinitesimal lattice limit is:
\[
    S^L = \beta\sum_s\frac{1}{4}g^2a^4F_{\mu\nu}(s)F_{\mu\nu}(s) + \mathcal O\left(a^5\right)
      \xrightarrow{a\to0} \beta g^2a^2\int\mathrm d^2x\,\frac{1}{4}F_{\mu\nu}(x)F_{\mu\nu}(x)
\]
Thus, if the $\beta$ parameter is set to $\beta \equiv 1/(g^2a^2)$,
the lattice action converges to the continuum action of Equation \eqref{eq:cont_action}.

\subsection*{Lattice topological charge}

The topological charge was defined in Equation \eqref{eq:top_charge_density} and \eqref{eq:top_charge}.
It depends linearly on the gauge field tensor, and,
considering the plaquette loop of Equation \eqref{eq:plaq},
it is clear that the complex argument of it provides the correct dependence.
Indeed:
\[
    \begin{aligned}
        \sum_s\arg U[\mathcal P_{12}(s)] &= \sum_s\arg e^{iga^2\left[F_{12}(s) + \mathcal O\left(a\right)\right]} \\
                                        &= ga^2\sum_s\left[F_{12}(s) + \mathcal O\left(a\right)\right] \\
                                        &\xrightarrow{a\to0} g\int\mathrm d^2x\,F_{12}(s) = 2\pi Q
    \end{aligned}
\]
The topological charge on the lattice can then be defined as:
\begin{equation}\label{eq:lat_top_charge}
    Q^L \equiv \frac{1}{2\pi}\sum_s\arg U[\mathcal P_{12}(s)]
\end{equation}
which is a sum of the local contributions:
\begin{equation}\label{eq:lat_loc_top_charge}
    q^L(s) \equiv \frac{1}{2\pi}\arg U[\mathcal P_{12}(s)]
\end{equation}
In the infinitesimal lattice spacing limit, $q^L(s)$ is equal to the topological charge of the plaquette $P_{12}(s)$:
\[
    q^L(s) \xrightarrow{a\to0} a^2q(s)
\]

%Finally, it remains to define $q^L(s)$, the lattice variant of topological charge density.
%In a discretized space, rather than defining a continuous density,
%it is more useful to define a local quantity associated with the most elementary two dimensional object (\ie a plaquette),
%which has an area of $a^2$.
%The local topological charge in the lattice should then converge to the topological charge of a plaquette in the infinitesimal lattice limit:
%This limit is verified with the following definition:
%\begin{equation}\label{eq:lat_loc_top_charge}
%    q^L(s) \equiv \frac{1}{2\pi}\arg U[\mathcal P(s)]
%\end{equation}
%The sum of all local topological charges returns the total topological charge, as it should be:
%\[
%    Q^L = \sum_sq^L(s)
%\]

This definition of the total lattice topological charge yields always integer values if implemented on a compact orientable manifold,
as it happens in the continuum theory.
This fact can be easily explained noticing that every local charge is proportional to the sum of the arguments of all link variables in the corresponding plaquette.
Furthermore, each link is crossed by two plaquettes in opposite directions, so its net contribution to the final sum vanishes.
The total charge is not zero because the argument of each plaquette is taken,
and every time the sum of its four link arguments is not in the interval $(-\pi,\pi]$,
the a shift of $2n\pi$ with $n\in\mathbb Z$ occurs.
The multiplicative factor of $1/(2\pi)$ in Equation \eqref{eq:lat_loc_top_charge} then make this shift to be the integer value $n$.

\subsection*{Monte Carlo simulations and continuum limit}

Equation \eqref{eq:cont_exp} gives the expectation value of an operator $\hat{\mathcal O}[A]$.
If it is possible to define a function $\mathcal O^L$ of Schwinger lines such that $\mathcal O^L[U] \xrightarrow{a\to0} \mathcal O[A]$,
the lattice variant of the expectation value will be:
\begin{equation}\label{eq:lat_exp}
    \left<\hat{\mathcal O}^L[U]\right>_L \equiv \frac{\int[\mathrm dU]\, \mathcal O^L[U] e^{-S^L[U]}}{\int[\mathrm dU]\,e^{-S^L[U]}}
\end{equation}
Here, the dynamical variables are the lattice Schwinger lines $U \in U(1)$, \ie all the link variables, and $\mathrm dU$ is the Haar measure of $U(1)$:
\[
    U = e^{i\varphi} \Longrightarrow \mathrm dU \propto \mathrm d\varphi
\]

It is useful to describe the problem in terms of \emph{probability theory} concepts:
$\mathcal O^L[U]$ can be seen as a function of random variables $\{U\}$ that follow a probability density function proportional to $e^{-S^L[U]}$,
and the mean value of $\mathcal O^L[U]$ corresponds to the expectation value of Equation \eqref{eq:lat_exp}.

To compute $\left<\mathcal O^L[U]\right>_L$, the most common strategy employs Monte Carlo algorithms.
The procedure is organized in these steps:
\begin{itemize}
    \item The physical system has to be implemented on a computer: link variables are stored in computer memory,
        and appropriate functions are defined to evaluate $S^L$ and $\mathcal O^L$ in terms of the links.
        This step will be discussed in Chapter \ref{ch:lattice}.
    \item A sequence of link configurations is generated in agreement with the weight function $e^{-S^L}$, and, at each step,
        $\mathcal O^L$ is evaluated at the current configuration, and stored in computer memory. The details will be explained in Chapters \ref{ch:mc} and \ref{ch:local}.
    \item The average of the stored values is computed, and, for the law of large numbers,
        it converges to the value of Equation \ref{eq:lat_exp}.
\end{itemize}

However, the lattice expectation value \eqref{eq:lat_exp} converges to the value in the continuum \eqref{eq:cont_exp} only in the limit of infinitesimal lattice spacing:
\begin{equation}\label{eq:cont_limit}
    \left<\hat{\mathcal O}^L[U]\right>_L \xrightarrow{a\to0} \left<\hat{\mathcal O}[A]\right>_E
\end{equation}
and this limit is not achievable on a computer, because it would imply to operate on an infinite number of links.
Two \emph{cutoffs} are then necessary: a finite lattice spacing $a$ (ultraviolet cutoff),
and a finite volume, \ie a finite number $N$ of lattice sites along every direction (infrared cutoff).
Usually, the finite volume is implemented with periodic boundary conditions,
because they preserve translational invariance, and hence they reduce the finite volume bias.

To recover the continuum limit and the infinite volume limit,
it is necessary to evaluate the expectation value with different parameters in order to extrapolate the limits.
In quantum field theories, this is in general a complicated problem,
since theory parameters have to be renormalized, and their value depends on the energy cutoff.

Letting the space be a finite sized cubic box with a side of length $L=Na$,
the continuum limit can be evaluated performing diffent simulations with different values of $N$, changing $a$ accordingly in order to maintain $L$ fixed.
In the considered theory, the action \eqref{eq:lat_action} depends on $a$ implicitly through the parameter $\beta = 1/(g^2a^2)$.

In general, if the coupling constant $g$ needs to be renormalized,
the renormalization group equations establish for $g$ a non-trivial dependence on the lattice spacing (\emph{running coupling constant}).
If the theory exhibits asymptotic freedom (like QCD, for example), the continuum limit dependence of $g(a)$ can be studied perturbatively, computing the $\beta_L$ function:
\[
    \beta_L(g) \equiv -a\frac{\partial g}{\partial a}
\]

However, for the two dimensional $U(1)$ gauge theory, the $\beta_L$ function is zero, and the coupling constant $g$ is then independent of the lattice spacing $a$ \cite{casana:1997}.
Thus, the action parameter $\beta$ dependence on $a$ is simply:
\[
    \beta(a) \propto a^{-2}
\]
In order to keep the side of the box $L$ constant, the product of $a$ and $N$ should be constant,
thus, the action parameter $\beta$ and the number of sites at each direction $N$ should follow the \emph{line of constant physics}:
\begin{equation}\label{eq:const_phys}
    \frac{\beta}{N^2} = \text{constant}
\end{equation}

To extrapolate the continuum limit of Equation \eqref{eq:cont_limit},
one then needs to evaluate different expectation values of Equation \eqref{eq:lat_exp},
one for each pair of parameters $(\beta,N)$, chosen on the line of constant physics of Equation \eqref{eq:const_phys}.
The continuum limit corresponds to the limit $1/\beta\to0$, and it can be extrapolated with a curve fit.

The more the pair of parameters are close to the continuum limit, the more the curve fit will be realistic,
since lattice artifacts will be more suppressed. 

As it will be shown in Chapter \ref{ch:local}, the topological freezing sets a threshold in how far it is possible to proceed toward the continuum with a local algorithm.
However, with the implementation of the cluster algorithm of Chapter \ref{ch:cluster}, this threshold will be removed.

The infinite volume limit will not be treated in this work
because it would not be useful to evaluate the cluster algorithm:
infinite volume extrapolations are performed at fixed lattice spacing and variable volume,
while the presence of topological freezing depends mainly on lattice spacing and not on volume.
Thus, the infinite volume extrapolation is not the main problem in presence of topological freezing.

%*****************************************
%*****************************************
%*****************************************
%*****************************************
%*****************************************

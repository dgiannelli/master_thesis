%************************************************
\chapter{Toy Model}\label{ch:toy_model}
%************************************************

The definition of the $U(1)$ pure gauge theory will follow.
The lagrangian, the path integral formulation and the topological charge will be firstly defined in the continuum, and then, on the lattice.

\section{Theory in the continuum}

In a pure gauge theory, the only dynamic fields are gauge fields.
Since $U(1)$ group has only one generator, there is only one gauge field $A_\mu$.

\subsection*{Minkowski metric}

In a $1+1$ dimensional Minkowski space-time, $A_\mu$ has two components, and $\mu = 0, 1$ are respectively the time and space indices.

The classic lagrangian of the theory is:
\[
    \mathcal L[A] = -\frac{1}{4} F_{\mu\nu}F^{\mu\nu}
\]
with $F_{\mu\nu}(x) \equiv \partial_\mu A_\nu(x) - \partial_\nu A_\mu(x)$.

The $U(1)$ local gauge transformation is a symmetry of this lagrangian, in fact,
under a transformation $G(x) \equiv e^{i\theta(x)} \in U(1)$,
the gauge field transformation is:
\begin{align*}
    A_\mu(x) \rightarrow A'_\mu(x) &= G(x) A_\mu(x) G^{-1}(x) - \frac{i}{g} G(x) \partial_\mu G^{-1}(x) = \\
                                   &= A_\mu(x) - \frac{1}{g} \partial_\mu \theta(x)
\end{align*}
where $g$ is the coupling constant.

The action is: \[ S[A] = \int \mathrm d^2x\, \mathcal L[A(x)] \] and the expectation value of an operator $\widehat{\mathcal O}[A]$ is:
\[
    \left< \widehat{\mathcal O}[A] \right> = \frac{\int[\mathrm dA]\, \mathcal O[A] e^{iS[A]}}{\int[\mathrm dA]\,e^{iS[A]}}
\]
where the integrals are path integrals over field configurations.
In this formulation, field configurations are multiplied by a complex exponential.
When the intent is to compute expectation values via Monte Carlo simulations,
it is necessary to sample configurations from probability density functions.
In this theory, the integrands can be made real and positive switching the theory to euclidean metric performing a Wick rotation of the time coordinate and an analytic continuation of expectation values.

\subsection*{Euclidean metric}

Indicating with the index $M$ the objects defined in Minkowski space, the Wick rotation:
\begin{align*}
    x^M_0 &\rightarrow -i x_2 & x^M_1 &\rightarrow x_1 \\
    A^M_0 &\rightarrow i A_2 & A^M_1 &\rightarrow A_1 
\end{align*}
induce the following transformation of the action:
\[
    S^M[A] = -\frac{1}{4}\int\mathrm d^2x^M\,F^M_{\mu\nu}F_M^{\mu\nu} \rightarrow  S[A] = \frac{i}{4}\int\mathrm d^2x\,F_{\mu\nu}F_{\mu\nu}
\]
Thus, the expectation value of an operator $\widehat{\mathcal O}[A]$ is:
\begin{equation}\label{eq:cont_exp}
    \left< \widehat{\mathcal O}[A] \right> = \frac{\int[\mathrm dA]\, \mathcal O[A] e^{-S[A]}}{\int[\mathrm dA]\,e^{-S[A]}}
\end{equation}
Here, the factor $e^{-S[A]}$ is real and positive, and can be enterpreted as a probability density function of field configurations.

The Minkowski formulation will not be needed anymore in later discussions.
For this reason, every object will be implied to be defined in euclidean two dimensional space.

\subsection*{Topological charge}

The gauge field, expressed in a generic gauge $G(x) \equiv e^{i\theta(x)} \in U(1)$ is:
\[
    A^G_\mu(x) \equiv A_\mu(x) - \frac{1}{g}\partial_\mu\theta(x)
\]

Let $S$ be a disk in the space, $\partial S$ its boundary and $R$ its radius.
If $R \to \infty$, the gauge field, evaluated on $\partial S$,
has only the contribution induced by the gauge transformation:
\[
    \lim_{R\to\infty}A^G_\mu(x) = -\frac{1}{g}\partial_\mu\theta(x)
\]
and its circulation integral over the border is:
\[
    \lim_{R\to\infty}\oint\limits_{\partial S}\mathrm dl_\mu\,A^G_\mu(x) = -\frac{1}{g}\oint\limits_{\partial S}\mathrm dl_\mu\partial_\mu\theta(x)
\]

The definition of $\theta(x)$ leaves a further unfixed constant term:
\[
    \theta(x) \rightarrow \theta(x) + 2\pi n, \quad \text{with}\ n \in \mathbb Z
\]
thus, being $\theta(x)$ a conservative function, only the constant term contributes to the circulation integral:
\[
    \lim_{R\to\infty}\oint\limits_{\partial S}\mathrm dl_\mu\,A^G_\mu(x) = \frac{2\pi}{g} Q, \quad \text{with}\ Q \in \mathbb Z \\
\]
where $Q$ is the topological charge.

Applying then Stokes theorem:
\begin{align*}
    Q &= \frac{g}{2\pi}\lim_{R\to\infty}\oint\limits_{\partial S}\mathrm dl_\mu\,A^G_\mu(x) = \\
      &= \frac{g}{2\pi}\lim_{R\to\infty}\int\limits_S\mathrm d\vv s\,\cdot \vv\nabla \times \vv A^G(x) = \\
      &= \frac{g}{2\pi}\lim_{R\to\infty}\int\limits_S\mathrm d^2x\, \left(\partial_1 A^G_2(x) - \partial_2 A^G_1(x)\right) = \\
      &= \frac{g}{2\pi}\lim_{R\to\infty}\int\limits_S\mathrm d^2x\, F^G_{12}(x) = \\
      &= \frac{g}{4\pi}\lim_{R\to\infty}\int\limits_S\mathrm d^2x\, \epsilon_{\mu\nu}F^G_{\mu\nu}(x)
\end{align*}
the definition of the topological charge density becomes clear:
\[
    q(x) \equiv \frac{g}{4\pi}\epsilon_{\mu\nu}F_{\mu\nu}(x)
\]
and its integral over all the space is the topological charge is:
\[
    Q = \int\mathrm d^2x\,q(x)
\]

Finally, being $\vv B(x) = \vv\nabla \times \vv A(x)$, the charge density is proportional to the flux of the magnetic field across an infinitesimal surface,
and the total charge is proportional to the total flux.

\section{Lattice formulation}
The goal is to compute physical observables numerically, and the expectation value \eqref{eq:cont_exp} provide a way to do that.
However, both the numerator and the denominator are divergent in the continuum, and it is necessary an integral regularization to isolate the divergent term,
and hence remove it.

The idea of lattice regularization is to define a theory on a discrete space time lattice with the requirement that,
when the lattice becomes finer, the lattice theory approaches to the continuum theory.

The path integrals are then approximated with finite dimensional integrals,
and the continuum expectation value \eqref{eq:cont_exp} is obtained as the continuum limit of the lattice approximated expectation value.

It is then necessary to provide new definition for the action and the topological charge on the lattice,
and they need to recover the continuum definition when the lattice spacing approaches zero.


\subsection*{













%*****************************************
%*****************************************
%*****************************************
%*****************************************
%*****************************************

%************************************************
\chapter{Conclusions and outlooks}\label{ch:conclusion}
%************************************************
\section{Conclusions}
The results of this work are here summarized:
\begin{itemize}
    \item
        An original local Metropolis-Hastings algorithm has been designed to sample link variables from a truncated Gaussian distribution.
        Laplace's method guarantees that the probability density function of $U(1)$ link variables converges to the truncated Gaussian
        as the $\beta$ parameter increases.
        The acceptance of this local algorithm (Figure \ref{fig:local_acc_cont}) has been very high for all parameters considered
        with values above 99\% for those that are close to the physical continuum limit.
    \item
        The topological freezing has been studied both numerically and visually.
        In particular, an original smoothing method of the topological charge has been introduced.
        It uses the image processing Gaussian blur algorithm, and, in this specific theory,
        it can be applied directly on local topological charge values instead of link variables.
        It is then easier to implement and visualize,
        as it is applied on matrices of real numbers instead of on $U(1)$ lattice link variables.
        The results it has produced are reported in Figures \ref{fig:charge_blur},
        \ref{fig:freezing}, \ref{fig:local_inv} and \ref{fig:cluster_inv}.
    \item
        An original cluster inversion algorithm has been defined.
        It selects a lattice square of arbitrary size, and sets each link variable included in it to its own complex conjugate.
        A chain of lattice local gauge transformation is applied to the border of the square before the cluster inversion:
        it makes real valued all border link variables, exept for one, making the energy variation of the entire move to be
        dependent of only one plaquette.
        The Metropolis acceptance of the algorithm is then of the same order of magnitude as the one due to a random local update.

        Alternating cluster inversion and sweeps of local updates, the entire configuration space can be explored effectively:
        the local updates are efficient in sampling Boltzmann distribution, and cluster inversions are efficient in charge tunneling.
    \item
        The same idea used to find an optimal local Metropolis-Hastings algorithm has been used to improve the performance of the cluster algorithm.
        Instead of conjugating all cluster links, the border link that is unfixed by the gauge transformation can be sampled from a truncated Gaussian distribution.
        In this way, the link is sampled from a distribution that is similar to the marginalized Boltzmann distribution, and hence, more likely to be accepted.
        However, since initial and final Gaussians are not the same, the acceptance is not as high as a local update,
        but it is still higher than the one obtained with the complete cluster inversion (Figure \ref{cluster_acc_cont}).

\end{itemize}


\section{Outlooks}

%*****************************************
%*****************************************
%*****************************************
%*****************************************
%*****************************************

%************************************************
\chapter{Local Algorithm}\label{ch:local}
%************************************************

The infrastructure is set up, and the Monte Carlo updating of link variables has to be defined.

In this chapter will be firstly introduced the Metropolis-Hastings algorithm,
and explained how it can be used to compute expectation values.

Then, a specific local algorithm will be defined to compute the integral of Equation \eqref{eq:lat_exp}.
The first simulation will be run,
and the result of the plaquette mean value will be compared with the value reported by D\"urr and Hoelbling in 2005 \cite{durr-hoelbling:2005}.

However, since the algorithm is local, at small lattice spacing, the topological freezing will be encountered.

\section{Markov chain Monte Carlo}

\subsection*{Markov chains and equilibrium distributions}
Let $x_1, x_2, x_3, \ldots \in \Omega$ be a sequence of random variables.
This sequence is a \emph{Markov chain} if the probability density function of the $(i+1)^\mathrm{th}$
variable depends only on the value of the $i^\mathrm{th}$ variable:
\[
    p(x_{i+1}|x_1, \ldots, x_i) = p(x_{i+1}|x_i) \quad \forall i\in\mathbb N^*
\]

In the following discussion will be only considered Markov chains that are:
\begin{itemize}
    \item \emph{Stationary}:
        \[
            p(x_{i+1}|x_i) = p(x_{k+i+1}|x_{k+i}) \quad \forall k\in\mathbb N^*
        \]
    \item \emph{Ergodic}:
        \[
            \forall x,y \in \Omega,\ \exists n\in\mathbb N^* : p(x_n=y|x_1=x) \neq 0
        \]
    \item \emph{Aperiodic}:
        \[
            \forall x,y \in \Omega,\ \nexists t\in\mathbb N^* :
            \begin{dcases}
                p(x_n=y|x_1=x)=0 \quad \forall n\neq t,2t,\ldots \\
                p(x_n=y|x_1=x)\neq0 \quad \forall n=t,2t,\ldots
            \end{dcases}
        \]
\end{itemize}

A Markov chain is univocally determided by its \emph{transition probability} from point $x$ to $y$, which is defined as:
\begin{equation}\label{eq:transition}
    w(x \to y) \equiv p(x_{i+1}=y|x_i=x)
\end{equation}

It is proven \cite{mc-mt} that there exists a unique probability density function $p(x\in\Omega)$
such that:
\begin{equation}\label{eq:equilibrium}
	p(x) = \int\mathrm dy\,w(y \to x)p(y)
\end{equation}
and $p(x)$ is called the \emph{equilibrium distibution} (or \emph{stationary distribution}) of the Markov chain.

\subsection*{Convergence of Markov chains}
The usefulness of Markov chains for statistical sampling is guaranteed by two important theorems:
the convergence theorem and the ergodic theorem. The proof of them can be found in \cite{mc-mt}.

The convergence theorem states that the probability density function of a variable that is $n$ steps forward in the Markov chain
converges to the equilibrium distribution $p(x)$, independently from the initial value $x_1$:
\begin{theorem}[Convergence Theorem]\label{th:convergence}
    \[
        \forall x_1 \in \Omega,\ p(x_{n}=x|x_1) \xrightarrow{n\to\infty} p(x)
    \]
    and the convergence rate is exponential:
    \[
        \sup_{x\in\Omega} |p(x_n=x|x_1) - p(x)| \stackrel{n\to\infty}{\scalebox{2}[1.25]{$\sim$}} e^{-\sfrac{n}{\tau_\mathrm{mix}}}
    \]
    where $\tau_\mathrm{mix}$ is the \emph{mixing time}.
\end{theorem}

The convergence theorem establishes a first connection between variables that are directly sampled from $p(x)$
and those which come from a Markov chain at equilibrium in $p(x)$.

This parallelism is completed with the ergodic theorem.
Let $f$ be a function of the random variable $x\in\Omega$ distributed according to $p(x)$.
The arithmetic mean of $f$ evaluated over a Markov chain in equilibrim at $p(x)$ converges to the mean value of $f(x)$:
\begin{theorem}[Ergodic theorem]\label{th:ergodic}
    \[
        \overline f_n \equiv \frac{1}{n}\sum_{i=1}^n f(x_i) \xrightarrow{n\to\infty} \left<f\right>
    \]
\end{theorem}

The ergodic theorem justifies the introduction of all the Markov chain machinery,
since it provides an alternative method to sample variables from a distribution and computing mean values over them.
Directly sampling is, in general, more efficient than Markov chain sampling,
but it is possible to directly sample only a very reduced class of distributions.
With Markov chain sampling, on the other hand, there are no significant restrictions, apart from efficiency.

In practical applications, the arithmetic mean of Theorem \ref{th:ergodic} is not performed over all values of the Markov chain,
and the first values are discarded.
This is justified by the fact that the first value of Markov chain is usually taken arbitrarily,
and it is therefore often very distant from the mean value and the region of typical fluctuations of the distribution.
The sequence takes then few iterations to get closer to the mean value, and, if these values are evaluated,
the precision of the mean value obtained will be heavily reduced.
The number of iterations necessary for the Markov chain to reach typical values of the distribution is called \emph{thermalization time}.
It is, in general, difficult to quantify, and the number of values to discard is usually chosen a posteriori, during the data analysis phase \cite{numerical_recipes}.


\subsection*{Autocorrelation}
However, Markov chain values are correlated,
and, even though this fact does not affect the unbiasedness of the average (Theorem \ref{th:ergodic}),
it has to be taken into account when choosing correct estimators of other quantities.%, such as the variance of the average $\overline f_n$, for example.

The autocorrelation function of $f(x\in\Omega)$ is:
\begin{align*}
    C_f(h) &\equiv \frac{(f(x_k)-\left<f\right>)(f(x_{k+h})-\left<f\right>)}{\sigma_f^2} \\
           &= \frac{\left<f(x_k)f(x_{k+h})\right>-\left<f\right>^2}{\sigma_f^2}
\end{align*}
and, using Theorem \ref{th:convergence}, it is possible to evaluate its asymptotycal behaviour.
In fact:
\begin{align*}
    \left<f(x_k)f(x_{k+h})\right> &= \int\mathrm dx_k\int\mathrm dx_{k+h}f(x_k)f(x_{k+h})p(x_k)p(x_{k+h}|x_k) \\
                                  &= \left<f\right>^2 + \mathcal O\left(e^{-\sfrac{h}{\tau_\mathrm{mix}}}\right)
\end{align*}
$C_f$ then decays exponentially with the number of forward Markov steps:
\begin{equation}\label{eq:autocorr_decay}
    C_f(h) \stackrel{h\to\infty}{\scalebox{2}[1.25]{$\sim$}} e^{-\sfrac{h}{\tau_\mathrm{mix}}}
\end{equation}

The first encounter with the autocorrelation function occurs when evaluating the variance of the average $\overline f_n$ of Theorem \ref{th:ergodic}:
\begin{align*}
    \sigma_{\overline f_n}^2 &= \left<\left(\frac{1}{n}\sum_if(x_i) - \left<f\right>\right)^2\right> \\
                             &= \frac{1}{n^2}\sum_{ij}\left<(f(x_i)-\left<f\right>)(f(x_j)-\left<f\right>)\right> \\
                             &= \frac{1}{n^2}\sum_{ij}\left(\left<f(x_i)f(x_j)\right>-\left<f\right>^2\right) \\
                             &= \frac{1}{n^2}\sum_i\left(\sigma_f^2+\sum_{j \neq i}\left<f(x_i)f(x_j)\right>\right)
\end{align*}
Using the symmetry $i \leftrightarrow j$ and the stationarity condition:
\begin{align*}
    \sum_i\sum_{j\neq i}\left<f(x_i)f(x_j)\right> &= 2\sum_i\sum_{j>i}\left<f(x_i)f(x_j)\right> \\
                                                  &= 2\sum_k\sum_{h=1}^{n-k}\left<f(x_k)f(x_{k+h})\right> \\
                                                  &= 2\sigma_f^2\sum_k\sum_{h=1}^{n-k}C_f(h)
                                                  %&\xrightarrow{n\gg\tau_\mathrm{mix}} 2\sigma_{\overline f_n}^2\sum_k\sum_{h=1}^\infty C_f(h)
\end{align*}
and using Equation \ref{eq:autocorr_decay}:
\[
    \sum_k\sum_{h=1}^{n-k}C_f(h) \xrightarrow{n\gg\tau_\mathrm{mix}} \sum_k\sum_{h=1}^\infty C_f(h) = n\tau_f^\mathrm{int}
\]
where $\tau_f^\mathrm{int} \equiv \sum_{h=1}^\infty C_f(h)$ is the \emph{integrated autocorrelation time} and
it is a good measure for the amount of autocorrelation of the Markov chain $f(x_1), f(x_2), \ldots$.
The result for the variance in the limit of a large sample size is then:
\begin{equation}\label{eq:variance}
    \sigma_{\overline f_n}^2 \xrightarrow{n\gg\tau_\mathrm{mix}} \sigma_f^2\frac{1+2\tau_f^\mathrm{int}}{n}
                             = \frac{\sigma_f^2}{n_f^\mathrm{eff}}
\end{equation}
where $n_f^\mathrm{eff} \equiv n/(1+2\tau_f^\mathrm{int})$ is the \emph{effective number} of uncorrelated values,
since the variance of the average for uncorrelated data would have been $\sigma_f^2/n$.

A good measure for the effectiveness of a Markov chain algorithm can then be the number of effective uncorrelated values produced in a certain amount of computer time,
and it can be used to choose simulation parameters, as it will be done in Chapter \ref{ch:results}.
The most efficient and simple way to evaluate $\sigma_{\overline f_n}^2$, $n_f^\mathrm{eff}$, and $\tau_f^\mathrm{int}$ is with the use of \emph{resampling methods}.
All the results present in this work are obtained with the \emph{binning} or the \emph{jackknife} resampling methods,
and the details of the implementation are showed in Appendix \ref{ap:data}.

\section{Metropolis-Hastings algorithm}
In the last section, it was showed how Theorems \ref{th:convergence} and \ref{th:ergodic} allow to sample a generic probability density function $p(x)$
with Markov chains that have $p(x)$ as their equilibrium distribution.
What is still missing in the discussion is how to define a Markov chain whose equilibrium distribution is exactly $p(x)$.

To define a Markov chain, it is sufficient to specify the transition probability of Equation \ref{eq:transition},
and the Metropolis-Hastings algorithm provides a way to do it.

\subsection*{Balance equations}
The equilibrium condititon of Equation \eqref{eq:equilibrium}
can be interpreted as a requirement of stability of the Markov chain.
In fact, using the normalization condition $\int\mathrm dy\,w(x \to y) = 1$,
it is equivalent to:
\[
	\int\mathrm dy\,w(x \to y)p(x) = \int\mathrm dy\,w(y \to x)p(y)
\]
which is the \emph{global balance} equation.
It states that the probability flux coming from point $x$
is equal to the probability flux going toward point $x$ from all other points $y$.



%*****************************************
%*****************************************
%*****************************************
%*****************************************
%*****************************************

%************************************************
\chapter{Local Algorithm}\label{ch:local}
%************************************************

The infrastructure is set up, and the Monte Carlo updating of link variables has to be defined.

In this chapter will be firstly introduced the Metropolis-Hastings algorithm,
and explained how it can be used to compute expectation values.

Then, a specific local algorithm will be defined to compute the integral of Equation \eqref{eq:lat_exp}.
The first simulation will be run,
and the result of the plaquette mean value will be compared with the value reported by D\"urr and Hoelbling in 2005 \cite{durr-hoelbling:2005}.

However, since the algorithm is local, at small lattice spacing, the topological freezing will be encountered.

\section{Metropolis-Hastings algorithm}

\subsection*{Markov chains and equilibrium distributions}
Let $x_1, x_2, x_3, \ldots \in \Omega$ be a sequence of random variables.
This sequence is a \emph{Markov chain} if the probability density function of the $(i+1)^\mathrm{th}$ variable depends only on the value of the $i^\mathrm{th}$ variable:
\[
	p(x_{i+1}|x_1, \ldots, x_i) = p(x_{i+1}|x_i), \quad \forall i
\]

The Markov chain is said to be \emph{stationary} if:
\[
	p(x_{i+1}|x_i) = p(x_{k+i+1}|x_{k+i}), \quad \forall i,k
\]
and $w(x \to y) \equiv p(x_{i+1}=y|x_i=x)$ is the \emph{transition probability} from $x$ to $y$.

A stationary Markov chain is at \emph{equilibrium} if there exist a probability density function $p(x \in \Omega)$ such that:
\begin{equation}\label{eq:equilibrium}
	p(x) = \int\mathrm dy\,w(y \to x)p(y)
\end{equation}
and $p(x)$ is the \emph{equilibrium distibution} of the Markov chain.

It is proven \cite{mc-mt} that if a Markov chain is irreducible and aperiodic,
the probability density function after a number $n$ of steps converges to the equilibrium distribution:
\begin{equation}\label{eq:convergence}
	p(x_{i+n}=x|x_i) \xrightarrow{n\to\infty} p(x)
\end{equation}
and the convergence rate is exponential:
\begin{equation}\label{eq:convergence_rate}
	\max_{x\in\Omega} |p(x_{i+n}=x|x_i) - p(x)| \xrightarrow{n\to\infty} 
\end{equation}

\subsection*{Balance equations}
The equilibrium condititon of Equation \eqref{eq:equilibrium}
can be interpreted as a requirement of stability of the Markov chain.
In fact, using the normalization condition $\int\mathrm dy\,w(x \to y) = 1$,
it is equivalent to:
\[
	\int\mathrm dy\,w(x \to y)p(x) = \int\mathrm dy\,w(y \to x)p(y)
\]
which is the \emph{global balance} equation.
It states that the probability flux coming from point $x$
is equal to the probability flux going toward point $x$ from all other points $y$.









%*****************************************
%*****************************************
%*****************************************
%*****************************************
%*****************************************
